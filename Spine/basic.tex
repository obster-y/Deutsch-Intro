\section{基础知识}

德语是一种属于日耳曼语族的使用基于拉丁字母表的全音素文字的屈折语。

这里讨论的是主要应用在德意志联邦共和国的德语(标准德语)。

\subsection{字母表}

基于拉丁字母表,德语增加了七个字母,ä,Ä,ö,Ö,ü,Ü,ß。其中位于元音上的两点称为变音符号(Umlaut),在不方便输入时可以通过在未变音元音后加e替代,如 ä -> ae。其中长得像希腊字母 beta 的「ß」并不是 beta,而是 ss 的连写,没有大写形式是因为它不会出现在词首,同时在任何情况下都可以被替换为 ss,同时发音一致。似乎在瑞士不使用这一符号。


\subsection{发音}


\subsubsection{辅音}
\subsubsection{元音}
\subsubsection{外来词的发音}
德语中的外来词主要来自于英语和法语,一些已经被完全同化了的词按照德语的发音规则发音,但不少词按照原始发音即可。
\subsection{书写习惯}

\subsection{基本句子结构}
