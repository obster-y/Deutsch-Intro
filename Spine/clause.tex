\section{复合句}

\subsection{并列句}
并列复合句(die Satzverbindung) 由两个独立的主句构成,即有两个独立的谓语,这两个句子通过连词(参考\cref{sec:conjunction})、具有连词功能的副词或少数情况无连词连接。需要注意的是,连词在句子中位置在前区之前(VVF),不占据前区位置,而使用副词需要占据前区。

\begin{itemize}
    \item 并列关系: und | auch(也), außerdem / ferner / zudem(此外), überdies(而且), ebenso / ebenfalls / gleichfalls(同样)

    \item 选择关系: oder | sonst, andernfalls

    \item 转折关系: aber | degegen, doch, hingegen, indes(sen), vielmehr

    \item 限制关系: allerdings, indes(sen), insofern, soweit, wohl, nur, zwar\ldots aber

    \item 原因关系: denn | nämlich

    \item 结果关系: also, daher, darum, demnach, demzufolge, deshalb, deswegen, folglich, infolgedessen, mithin, so, somit

    \item 让步关系: dennoch, des(sen), ungeachtet, gleichwohl, immerhin, tortzdem, nichtstdestow-eniger, nichtsdestotrotz
\end{itemize}

\subsection{从句的基本知识}

主从复合句(die Subordination) 是由主句(die Hauptsatz)和从句(die Nebensatz)组合而成,从句是主句的一个句子成分,没有独立的谓语。

\subsubsection{从句中的语序关系}

在从句中,句框和主句不再相同。LSK是从句的引导词,RSK是从句的谓语,MF是从句的主体,没有VF和NF。而在RSK中,变位动词位于非变位动词后,同时可分前缀不再分开。

如果 LSK 有引导词,则从句谓语位于RSK,包括变位动词或者不定式。例外是,在由 als 引导的非现实从句(tut so, als\ldots)中,变位动词和 als 位于 LSK。

\example{Er sagt, \ult{dass}{NLSK} er ein Zimmer für uns \ult{reserviert}{NRSK}.}{他说他给我们订了一间房。}

\example{Es ist besser, \ult{die Wahrheit}{NLSK} \ult{zu sagen}{NRSK}.}{最好说出事实真相。}

\example{Er tut so, \ult{als könnte}{NLSK} er nicht bis drei \ult{zählen}{NRSK}.}{他装作什么都不懂的样子。}

如果LSK没有引导词,则谓语占据 LSK,如果是宾语或主语从句,变位动词位于从句主语后;如果是状语从句,变位动词位于句首。

\example{Er sagt, er \ult{reservier}{NLSK} ein Zimmer für uns.}{他说,他给我们订了一间房。}

\example{\ult{Sagst}{NLSK} du auch die Wahrheit, keiner wird dir glauben.}{就算你说出真相也没有人会相信你。}

\subsubsection{由从属连词引导}

\example{Ich \ult{glaube}{LSK}, \ult{dass}{NLSK} er hiergeblieben \ult{ist}{NRSK}.}{我想他待在这里了。}

\example{Ihr \ult{müsst}{LSK} das Klassenzimmer \ult{aufräumen}{RSK}, \ult{bevor}{NLSK} der Lehrer \ult{kommt}{NRSK}.}{你们必须在老师来之前把教室收拾好。}

\subsubsection{从句和主句的位置关系}

\begin{itemize}
    \item 从句位于主句前: 主语、宾语、状语、表语从句,都可以位于主句之前。
    
    \example{\uline{Dass ihr auch heir seid}, freut uns}{你们都到了,我很高兴。}
    
    \example{\uline{Wie er ist}, wird er immer bleiben.}{他会一直像现在这样。}
    
    \item 从句位于主句中: 即从句位于 MF,主要在状语从句中出现
    
    \example{Er ging, \uline{obwohl er eine Grippe hatte}, im kalten See schwimmen.}{尽管他得了感冒,但他在冰冷的湖水中游泳。}

    \example{Ihr solltet jetzt wegfahren, \uline{weil ihr}, \uline{wenn ihr noch lange wartet}, im Stau stehen werdet.}{你们现在就该出发了,因为,如果你们继续等带下去的话,就会遇到堵车。}

    \item 从句位于主句后: 基本操作
\end{itemize}

\subsubsection{由代词引导}
\begin{itemize}
    \item 由关系代词

    \example{Ist das der Minirock, den du gestern gekauft hast?}{这就是你昨天买的超短裙吗?}

    \example{Es gibt noch einiges, was ich nicht verstehe.}{还有一些我不明白的地方。}

    \item 由疑问代词/疑问副词
    
    \example{Ich weiß nicht, wie er heißt.}{我不知道他叫什么。}

    \example{Darf ich Sie fragen, warum Sie für Germanistik entschieden haben?}{我能问您一下,您为什么选择日耳曼文学专业?}

    \item 由含代词的介词词组

    \example{Das ist der Mann, von dem ich dir erzählt habe.}{这就是我跟你说过的那个男人。}

    \example{Die Professoren, mit denen wir uns treffen wollen, haben leider keine Zeit.}{我们要见面的教授可惜没有时间。}

    \item 代副词

    \example{Er hat alles erreicht, wovon er geträumt hat.}{他实现了他所有的梦想。}

    \example{Wir haben gestern einen schönen Kulturabend erlebt, worüber wir uns sehr freuen.}{我们昨天参加了一场美妙的文艺晚会,对此我们非常高兴。}
\end{itemize}
\subsubsection{无引导词的从句}

无引导词的从句,实际上大多数是从有引导词的复合句转换而来,主要出现在宾语从句/条件从句/让步从句中。通常表达猜测、陈述和想法

\begin{itemize}
    \item 宾语从句: 主要指以dass引导的宾语从句,去掉dass,且通常为第一虚拟式。

    \example{Er sagt, \uline{dass} er keine Lust dazu \uline{hat}.\\Er sagt, er \uline{habe} keine Lust dazu.}{他说,他对此没有兴趣。}

    \item 条件从句: 主要指以 wenn 或 falls 引导的从句中,去掉连接词。
    
    \example{\uline{Wenn} Sie heute keine Zeit \uline{haben}, brauchen Sie uns nicht zu besuchen.\\\uline{Haben} Sie heute keine Zeit, brauchen Sie uns nicht zu besuchen.}{你们没有时间的话就不用到我们这来了。}

    \example{\uline{Falls} Sie noch Fragen \uline{haben}, können Sie sich an unseren Schulleiter wenden.\\\uline{Haben} Sie noch Fragen, können Sie sich an unseren Schulleiter wenden.}{你们如果还有问题,可以咨询我们的校长。}

    \item 让步从句: 主要指以 obwohl, obgleich, trotzdem, wenn auch 引导的从句中,去掉连接词,主句添加 doch/trotzdem/dennoch, 从句添加 auch。
    
    \example{\uline{Obwohl} sie unschuldig war, wurde sie verurteilt.\\War sie \uline{auch} unschuldig, sie wurde \uline{doch} verurteilt.}{虽然她没有错,但被判有罪。}

    \example{Die beiden Schwestern sind sehr verschieden, \uline{obgleich} sie Zwillinge sind.\\Sind sie \uline{auch} Zwillinge, die beiden Schwestern sind \uline{doch} verschieden.}{两姐妹虽然是双胞胎,但却大不一样。}
\end{itemize}

\subsubsection{不定式结构}

不定式结构的主要不同是,不定式在形式上没有主语,意义上主语为主句的主语或者逻辑主语。无人称的复合句中,意义上的主语为 man。

\example{Die Eltern erlauben ihm, dass er ein Skateboard kauft.\\\indent
$\hookrightarrow$Die Eltern erlauben ihm, ein Skateboard \uline{zu kaufen.}}{父母同意他买滑板。}

\example{Er empfiehlt mir, dass ich das Gericht probiere.\\\indent
$\hookrightarrow$Er empfiehlt mir, das Gericht \uline{zu probieren}.}{他推荐我品尝这道菜肴。}

\example{Es ist unmöglich, \uline{dass man} diese Arbeit innerhalb von einem Tag schafft.\\\indent
$\hookrightarrow$Es ist unmöglich, diese Arbeit innerhalb von einem Tag \uline{zu schaffen}.}{在一天之内完成这项工作是不可能的。}

\subsubsection{分词结构}

分词结构主要用于书面语中,分词结构在形式上没有主语,意义上主语为主句的主语或者逻辑主语。

第一分词所表达的动词发生的时间与主句动词具有同时性,并且由主动的意义。不及物动词的第二分词发生的时间先于主句的时间,同样表示主动含义。及物动词的第二分词主要表示被动的含义,但对于主动/被动性不是特别强调,需要根据语境判断时间。

\example{Aus dem Gefängnis entlassen, überfiel er gleich eine Bank.\\\indent
$\hookrightarrow$Nachdem er aus dem Gefängnis entlassen worden war, überfiel er gleich eine Bank.}{他刚从监狱释放出来就抢劫了一家银行。}

\example{Weil die Frau eine Erkältung befürchtete, zog sie den Pelzmantel an.\\\indent
$\hookrightarrow$Eine Erkältung \uline{befürchtend}, zog die Frau den Pelzmantel an.}{这位女士由于害怕感冒,穿上了皮大衣。}

\example{Weil sie mit dem Bus verspätet angekommen sind, haben die Leute den Anfang des Konzerts versäumt.\\\indent
$\hookrightarrow$Verspätet mit dem Bus \uline{angekommen}, haben die Leute den Anfang des Kos Konzerts versäumt.}{由于他们乘坐公交车晚点到达,错过了音乐会开始的部分。}

\example{Nachdem er aus dem Gefängnis entlassen worden war, überfiel er eine Tankstelle.\\\indent
$\hookrightarrow$Aus dem Gefängnis \uline{entlassen}, überfiel er eine Tankstelle.}{他从监狱被释放后,便袭击了一个加油站。}


无人称分词结构一般由副词和第二分词构成。如 kurz gesagt, allgemein gesprochen, streng genommen, damit verglichen, anders formuliert, sehr mild ausgedrückt, so betrachtet, grob geschätzt。同时第一分词也可以构成无人称分词结构,可以转换成 wenn 引导的从句,分词短语的主语为不定代词 man。

\example{Wenn man es streng nimmt, ist diese Lösung nicht richtig.\\\indent
$\hookrightarrow$\uline{Streng genommen} ist diese Lösung nicht richtig.}{严格来说这个答案是不正确的。}

\example{Er ist, wenn man es milde ausdrückt, ein Trottel.\\\indent
$\hookrightarrow$Er ist, \uline{milde ausgedrückt}, ein Trottel.}{说得轻一点,他是个笨蛋。}

\subsection{从句的功能}


\begin{longtblr}[
    theme=nocaption,
    presep={0pt},
]{
    width=\linewidth,
    rowhead=1,
    colspec={clX[l]},
    columns={l,colsep=3pt},
    column{1}={wd=1em},
    rows={m,rowsep=0.2pt},
    cell{1,25,26}{1}={c=2}{l},
    cell{2}{1}={r=5}{c,m},
    cell{7}{1}={r=6}{c,m},
    cell{13,18}{1}={r=5}{c,m},
    cell{23}{1}={r=2}{c,m},
}
    \textbf{语法功能与类型} & & \textbf{例句} \\
    \hline
    主语从句 & dass 引导 & {\uline{Dass} du recht hast, ist zweifellos.\\你毫无疑问是有道理的。} \\
    & ob 引导 & {\uline{Ob} ich meinen Schüssel wiederfinde, ist noch fraglich.\\我是否能找回钥匙还是个问题。} \\
    & W- 引导 & {\uline{Wie} wir zum Bahnhof fahren, hängt von dir ab.\\我们怎么去火车站由你决定。} \\
    & 不定式引导 & {So viele Vokabeln \uline{zu beherrschen}, fällt mir schwer.\\掌握这么多单词对我挺难。} \\
    & 无连接词 & {Es scheint ihm, es werde immer dunkler.\\他觉得天色越来越黑。} \\
    \hline
    宾语从句 & dass 引导 & {Ich weiß, \uline{dass} ihr euch gut versteht.\\我知道,你们相处得很好。} \\
    & ob 引导 & {Ich zweifle daran, \uline{ob} er wirklich krank ist.\\我怀疑他是不是真的病了。} \\
    & W- 引导 & {Sie fragt mich, \uline{was} für einen Wunsch ich habe.\\她问我有一个什么样的愿望。} \\
    & 不定式引导 & {Ich hoffe, bald gesund \uline{zu werden}.\\我希望尽快恢复健康。} \\
    & 无连接词 & {Er sagt, er komme nicht zum Unterricht.\\他说他不来上课了。} \\
    & Wie 引导 & {Ich fühle, \uline{wie} sie mich liebt.\\我感受到她是多么爱我。} \\
    \hline
    状语从句 & 时间状语 & {\uline{Seitdem} er in einer anderen Firma arbeitet, hat er keinen Kontakt mehr mit un.\\自从他在另一家公司上班,他就跟我们没有联系了。} \\
    & 地点状语 & {Ich begleite dich, \uline{wohin} du wilst.\\我陪你去任何你想去的地方。} \\
    & 情状状语 & {Er finanziert sich selbst, \uline{indem} er auƪerhalb des Studiums noch Geld durch Teilzeitjobs verdient.\\他在经济上独立自主,在学习之余兼职打工挣钱。} \\
    & 原因状语 & {\uline{Weil} er die Prüfung nicht besteht, muss er sie nachholen.\\因为他没通过考试,必须补考。} \\
    & 条件状语 & {\uline{Wenn} du Lust hast, können wir einen Spaziergang machen.\\你要是有兴趣,就跟我们去散步吧。} \\
    状语从句 & 让步关系 & {Er nickt immer den Kopf, \uline{obwohl} er nichts versteht.\\他总是点头,尽管他什么也没听懂。} \\
    & 结果关系 & {Sie hat die ganzen Ferien gefaulenzt, \uline{so dass} sie die Hausarbeit nicht rechtzeitig abgeben kann.\\她懒惰了一整个假期,以至于不能按时上交家庭作业。} \\
    & 目的关系 & {Die Studenten müssen genug Kreditpunkte haben, \uline{damit} sie ihr Studium abschließen können.\\学生们必须修满足够的学分才能毕业。} \\
    & 对比关系 & {Ich habe den ganzen Tag gearbeitet, \uline{während} du dabei nur ferngesehen hast.\\我工作了一整天,而你却在一旁看电视。} \\
    & 不定式引导 & {In Paris \uline{angekommen}, suchten wir ein Hotel.\\到巴黎之后,我们就去找宾馆。} \\
    \hline
    定语从句 & 关系从句 & {Ich gehe auf die Grammatik, die wir gestern schon besprochen haben, nochmal ein.\\我再讲一下我们昨天讲过的语法。} \\
    & 不定式结构 & {Der Fußballer, in China geboren, wurde in Deutschland ein Star.\\这位在中国出生的足球运动员在德国是个明星。} \\
    \hline
    表语从句 & & {Sie bleibt, wie sie ist.\\她总是这个样子。} \\
    \hline
    延伸从句 & & {Morgen findet die F1 statt, worauf ich sehr gespannt bin.\\明天举行F1汽车赛,我非常期待。}
\end{longtblr}

\subsubsection{宾语从句}
宾语从句也被称为「内容从句」(Inhaltssätze),有几种不同的表达形式。

\subsubsection*{dass 引导}

并非所有主句可以连接 dass 引导的从句。通常来说,如果主句中含有表达情感、感觉、意愿、推测或表达想法等意义的动词时,经常与 dass 连接;或受某些名词化的动词或形容词支配。如果主句和从句具有相同的主语,dass 可转换为不定式结构。同时,dass 被广泛应用于间接引语。

\example{Ich befürchte, dass du Unrecht hast.}{我担心你是错的。}

\example{Die Gewissheit, dass wir in zwanzig Minuten ankommen.}{确定我们二十分钟后到达。}

\example{Ich habe die Gewissheit, \uline{dass} ich die Aufgabe bewältigen kann.\\\indent
$\hookrightarrow$Ich habe die Gewissheit, die Aufgabe bewältigen \uline{zu können}.}{我确信能够完成任务。}

\example{Sie flüsterte, es sei besser, dem Chef nicht zu widersprechen.\\\indent
$\hookrightarrow$Sie flüsterte, dass es besser sei, dem Chef nicht zu widersprechen.}{她小声说,最好不要反驳老板。}

\subsubsection*{ob 引导}

ob 引导的从句表达不稳定/疑虑的意思,针对 ob 从句可以用 J/N 疑问句提问,这样从句相当于一个转述的补充疑问句,从句中的变位动词可以是直陈式或虚拟式。有时不能直接把从句转换成疑问句,需要其他的句子成分。ob从句还可以表达怀疑和疑虑,如果要表达「选择」的含义,从句后常有 oder nicht。ob 表达疑虑/疑惑时也可受名词化后的动词/形容词的支配。

\example{Ich erkundigte mich: „Ist Ihre Frau wieder gesund?"\\\indent
$\hookrightarrow$Ich erkundigte mich, ob seine Frau wieder gesund sei/war.}{我打听他的太太是否恢复了健康。}

\example{Der Schalterbeamte weiß die Antwort auf die Frage: „Fährt heute noch ein Zug?"\\\indent
$\hookrightarrow$Der Schalterbeamte weiß, ob heute noch ein Zug fährt.}{窗口的工作人员知道今天是否还有火车。}

\example{Er zweifelt, ob er sie einladen soll.}{他在考虑,是否应该邀请她。}

\example{Es ist wichtig, ob du die Wahrheit sagst (oder nicht)}{你是否说出真相很重要。}

\example{Die Mitteilung, ob er an der Tagung teilnimmt.}{他是否参加大会的消息。}

\subsubsection*{W- 引导}

\begin{itemize}
    \item 疑问代词: wer, was, welcher, was für ein
    \item 介词+疑问代词: mit wem, für wen, über welchen
    \item 疑问副词: wo, wann, wie, weshalb, wie oft
    \item 疑问代副词: wovon, womit, \ldots
\end{itemize}

\example{Er dachte nach: „Wen soll ich einladen?"\\\indent
Er dacte nach, wen er einladen solle.}{他在想他应该邀请谁。}

\subsubsection{状语从句}

状语从句修饰的是句子的谓语。

\subsubsection*{时间状语的时间顺序}

从句事件的时间,在早于主句、同时于主句、晚于主句时,可以分为同时性(die Gleichzeitigkeit)、先时性(die Vorzeitigkeit)和后时性(die Nachzeitigkeit)。
\begin{table}[h]
    \centering
\begin{tblr}{
    % rowhead=1,
    colspec={clX[l]},
    columns={l,colsep=3pt},
    column{1}={wd=1em},
    rows={m,rowsep=1.50pt},
    cell{2}{1}={r=6}{m},
    cell{8}{1}={r=5}{m},
    cell{13}{1}={r=3}{m},
}
    & \textbf{引导词} & \textbf{例句} \\
    \hline
    同时性 & {während, indes, indessen\\主句动作在从句的时间框架内} & {Das Haus ist abgebrannt, während sie im Kino waren.\\他们看电影的时候房子着火了} \\
    & {solange\\主从句时间具有相同起止点} & {Du kannst bleiben, solange du willst.\\你想待多久就待多久。}\\
    & {seit, seitdem\\主从句起点,持续时间相同} & {Er ist viel ausgeglichener, seit er nich mehr trinkt.\\自从他戒了酒之后,他平和了许多。} \\
    & {wenn, sobald, sowie\\主从句发生在将来} & {Wenn du fertig bist, darfst du gehen.\\你做完了就可以走了。}\\
    & {als\\主从句在过去,一次性动作} & {Wir besucheten euch, als die Ferien zu Ende waren.\\假期结束时,我们拜访了你们。} \\
    & {soft, wenn\\从句重复性动作} & {Wenn er seine kleine Nichte sah, freute er sich.\\他每次看到他的小侄女都很高兴。} \\
    \hline
    先时性 & {nachdem, wenn\\从句发生在现在或将来} & {Nachdem wir die Arbeit erledigt haben, gehen wir nach Hause.\\我们完成工作之后就回家} \\
    & {nachdem, als\\从句发生在过去} & {Sie durften erst draußen spielen, nachdem sie die Huasaufgaben gemacht hatten.\\他们在完成家庭作业之后才可以出去玩。} \\
    & {solbad, sowie\\主句紧接从句发生} & {Wir werden euch besuchen, sobald wir in Linz angekommen sind.\\我们一到达 Linz 就去找你们。} \\
    & {wenn\\从句重复性动作} & {Er ist betrunken, wenn, er in der Kneipe gewesen ist.\\他去了酒吧就会喝醉。} \\
    & {seit, seitdem\\从句开始时间早于主句} & {Sie gingen häufig ins Kino, seit sie in die Stadt gezogen waren.\\自从他们搬到了城里,就经常去看电影。} \\
    \hline
    后时性 & {bevor, ehe\\主从时态相同,从句靠后} & {Lesen Sie die Gebrauchsanweisung, bevor Sie das Gerät benutzen.\\使用仪器之前请您先阅读使用说明书。} \\
    & {bis\\时间的终结} & {Bis der Krieg anfing, war seine Kindheit glücklich.\\他的童年在战争爆发之前还是幸运的。} \\
    & {bevor, ehe, bis\\有条件的否定} & {Ehe ich nich mit allen Parteien gesprochen habe, treffe ich keine Entscheidungen.\\在没有和各方谈论之前,我不做任何决定。} \\
\end{tblr}
\end{table}

\subsubsection*{情状从句}

\begin{itemize}
    \item (伴随) indem, dadurch\ldots dass: 表明主句动作以何种方式进行

    \example{Er hat seinen Wut dadurch ausgedrückt, dass er seine Freundin schimpfte.}{他骂他的女朋友来出气。}

    \item (伴随) wobei: 主从伴随,即主句行为是在从句动作伴随下发生的

    \example{Die Servierin fordert die Gästen auf hereinzukommen, wobei sie auf einen Platz am Fenster zeigt.}{服务员一边请客人进来,一边指着靠窗的一个位子。}

    \item (伴随) ohne dass, ohne zu: 即主句行为是在缺少从句动作伴随下发生的

    \example{Ihr habt uns geholfen, ohne dass wir euch darum gefragt haben.}{没等我们开口,你就帮了我们。}

    \item (伴随) (an)statt dass, (an)statt zu: 取舍伴随,即舍弃了从句的动作,选取了主句的动作,如果主语一致,可以使用 zu

    \example{Anstatt dass der Güterverkehr auf die Schiene verlagert wird, baut man neue Straßen.}{货物运输的重心不再是铁路,而是人们为此兴建的新的公路。}

    \item (限制) insoweit, soweit, soviel: 据\ldots 了解/所知,从句限制了主句动作的发生

    \example{Soviel ich informirt bin, hat niemand von uns die Prüfung bestanden.}{据我所知我们都没通过考试。}

    \item (限制) als, insofern als, insoweit als: 在\ldots 的范围之内,就这点而言,主句的行为要在从句给定的条件内才有效。

    \example{Ihr habt insofern Recht, als euer Lösungsweg ebenfalls möglich ist.}{你们的解决方案是可行的,就这点来看你们是有道理的。}

    \item (限制) außer dass, außer wenn: 除\ldots 之外,从句给予主句一定的限制

    \example{Ich habe am Sonntag nichts getan, außer den ganzen Tag zu lesen.}{我周日除了读了一天的书之外,什么事都没干。}

    \item (比较) wie, als: 一般针对主句中的形容词或副词进行比较(肯定/否定形式),主句常有关联词 so, ebenso, genauso 或 gleich;如果主从句动词一致,则从句主语与逗号可以省略;wie 表示同级,als 表示不同级;在口语中,可以用 wie 代替 als 表达不同级意义
    
    \example{Heute ist es wirklich \uline{so kalt}, wie es berichtet wurde.}{今天的天气真的跟预报的一样冷。}

    \example{Der Film ist \uline{besser}, als ich dachte.}{这部电影比我想象的要好。}

    \example{Dieses Winter ist \uline{nicht so kalt}, wie es in den letzten Jahren war.}{这个冬天不像以往那样冷。}

    \example{Ihr dürft nicht \uline{länger} bleiben, als wir es euch er laubt haben.}{你们在这里逗留的时间不能超过我们所允许的时间。}

    \example{Ich habe heute \uline{nicht so viel} gekocht wie gestern.}{我今天不像昨天似的,做了那么多饭。}

    \example{Ich habe \uline{anders} gedacht als er (denkt).}{我的想法跟他不同。}

    \example{Er ist \uline{größer}, wie sein Vater es war.}{他比他父亲还要高。}

    \item (非现实比较) als ob, als wenn, wie wenn, als: 表示与实际不符的过程或状况,或对现实错误的估计和预测。主句可以使用关联词 so,从句中的变位动词应为第二虚拟式。如果是由 als 引导的,则变位动词位于 als 之后
    
    \example{Er sieht mich \uline{so entgeistert} an, als ob er alles selbst erlebt hätte.}{他讲的如此生动就像他亲身经历了一样。}

    \example{Der Hund knurrte \uline{gefährlich}, als wenn er gleich zubeißen würde.}{狗发出危险的吠叫声,好像马上要咬人一样。}

    \example{Er gibt das Geld \uline{aus}, als wäre er Millionär.}{他花起钱来想百万富翁似的。}

    \item (比例关系) je\ldots desto, je\ldots umso, je\ldots nachdem: 其中 je\ldots desto, je\ldots umso对主从句中不断变化的过程或状态进行比较,即「越\ldots 越\ldots 」,其中「je + 形容词比较级」引导从句,变位动词位于从句末,「desto / umso + 形容词比较级」引导主句,变位动词位于第二位,之后是主语。je\ldots nachdem引导的句子,主句提出两个结构,由从句提出的条件取舍,即「依\ldots 而定」。有时从句所给的条件决定了对主句陈述事实的取舍,也可以用 abhängig von\ldots sein 改写。

    \example{\uline{Je mehr} Seiten eine Zeitung hat, \uline{desto mehr} Informationen gibt sie.}{一份报纸的张数越多,所含信息量就越大。}

    \example{\uline{Je fleißiger} er studiert, \uline{umso mehr} Forschritte macht er.}{他越勤奋地学习,进步就越大。}

    \example{Die Vortragsreihe wird im September oder Oktober beginnen, \uline{je nachdem}, wann der Professor von seiner Auslandsreise zurückkehrt.}{系列讲座将于九月或十月开始,需要看教授什么时候从国外回来了。}

    % \example{\uline{Je nachdem}, ob es eine günstige Zugverbindung gibt, werden wir mit dem Auto oder mit dem Zug fahren.}{我们是开汽车还是坐火车要看有没有方便的火车线路了。}

    \example{\uline{Je nachdem}, ob wir an die See oder in Gebirge fahren, müssen wir Badesachen oder eine Wanderausrüstung mitnehmen.\\
    $\hookrightarrow$\uline{Abhängig davon}, ob wir an die See oder ins Gebirge fahren, müssen wir Badesachen oder eine Wanderausrüstung mitnehmen.}{我们是带游泳装备还是徒步装备取决于我们是去海边还是山区。}
\end{itemize}

\subsubsection*{原因从句}
原因从句解释主句动作的起因、缘由

\begin{itemize}
    \item weil, da: 通常可以互换,其中 weil 常见于日常口语,da 常见于书面语

    \example{Die Nachbarin steht am Fenster, weil sie sehr neugierig ist.}{邻居站在窗前,因为她很好奇。}

    \example{Da ich seit Jahren in der Stadt wohne, macht mir der Verkehrslärm nichts aus.}{因为我在城市居住好多年了,噪声对我来说已经没什么了。}

    \item 只能使用 weil: 回答 warum, weshalb, wieso 时;列举重要原因或之前没提出过新内容时;主句含有 daher, dahrum, deshalb, deswegen, aus dem Grund 等表示强调的关联词时;
    
    \example{-Warum kommst du jetzt?\\    -Weil der Zug Verspätung hatte.}{-你为什么现在才来?\\-因为火车晚点了。}

    \example{-Warum trinkst du nur einen Kaffee?\\-Ich nehme nur einen Kaffee, weil ich schon gegessen habe.}{-你为什么只喝一杯咖啡?\\-我只喝一杯咖啡,是因为我吃过饭了。}

    \example{Ich habe schon gegessen. Da ich jetzt keinen Hunger mehr habe, nehme ich nur einen Kaffee.}{我已经吃过饭了。因为我现在不饿,所以我只点了一杯咖啡。}

    \example{Das Haus stürzte darum ein, weil sämtliche Bauvorscriften missachtet worden waren.}{这栋房子之所以倒塌了,是因为忽视了建筑规定。}

    \item 如果weil后面成分复杂,从句变位动词可以位于LSK
    
    \example{Ich habe ihn nicht gleich erkannt, weil ich \uline{habe} ihn lange nicht gesehen und er hat sich außerdem in den Jahren sehr verändert.}{我没一下子认出他来,因为我们很长时间没见面了而且他在这几年的变化很大。}
    
    \item 只能使用 da: 在给出的原因在前文提到或上下文或语境可以推测出时,尤其是居中含有 ja, bekanntlich 等词时
    
    \example{Da (ja) die Sonnenenergie eine sanfte Energieform ist, lässt sie sich nur schwer speichern.}{由于太阳能是一种温和能源,所以难于存储。}
    
    \item zumal, umso mehr als: 含有强调从句表示的理由的含义,一般为附加理由,只能在主句之后,可以翻译为「更是因为\ldots 还因为\ldots」,句中一般重读
    
    \example{Ich hole gerne die Karten für dich ab, zumal (da) ich heute Nachmittag sowieso in die Stadt fahre.}{我很愿意给你取票,而且我今天下午反正也要进城一趟。}

    \example{Weil unsere Freunde abgesagt haben, fuhren wir nicht zum Segeln, umso mehr als Sturm angesagt war.}{我们的朋友取消了计划,所以我们就不去驾帆船了,而且今天预报说有暴风。}
\end{itemize}

\subsubsection*{条件从句}
条件从句用来表达主句动作发生的条件。

\begin{itemize}
    \item wenn: wenn 的含义包括了 falls 和 sofern, wenn 引导的从句一般位于主句前,主句中有关联词 so dann。有时时间从句和条件从句不容易区分,它们主要的区别在于回答 wann 或者 Unter welcher Bedingung,以及 dann 可以出现在两种从句的主句中,但 so 只能出现在条件从句的主句中。

    \example{Wenn der Zug pünktlich ankommt, so / dann erreichen wir den Anschlusszug.}{如果火车正点到达,我们还可以赶得上转乘的火车。}

    \example{Wenn das Haus fertig ist, dann können wir einziehen.\\$\hookrightarrow$(条件从句)Unter der Beidingung, dass das Haus fertig ist, dann können wir einziehen.\\$\hookrightarrow$(时间从句)An dem Tag, an dem dass Haus fertig ist, dann können wir einziehen.}{如果房子建成,我们就可以入住了。\\当房子建成时,我们就可以入住了。}

    \example{Wenn das Haus einen Keller hat, so wird es wesentlich teurer.}{如果房子有地下室,那就会贵很多。}

    \item falls: 与 wenn 意义相同,且只能引导条件从句

    \example{Falls sie nach Berlin kommt, will sie das Brandenburger Tor besuchen.}{要是他去柏林,他会参观 Brandenburger 门。}

    \item sofern: 与 wenn, falls 意义类似,但更强调条件的必要性和先决性,且针对个别案例所提出的可能出现的条件,从句位置可以在主句前或主句后。

    \example{Sofern ihr mir zugehört habt, wisst ihr, was ihr tun müsst.}{你们只要仔细听我的,就知道该做什么了。}

    \example{Sie werden zusammenarbeiten, sofern man ihnen Verständnis entgegenbringt.}{只要双方表示出理解,就会走向合作。}

    \item 无连接词引导,此时从句总在主句前,从句变位动词在 LSK,主句往往有关联词 so, dann

    \example{Kommst du morgen nicht, ruf mich bitte vorher an.}{如果你明天不来,就提前给我打电话。}

    \example{Gefällt dir die Ware nicht, dann kann sie innerhalb von drei Tagen zurückgegeben werden.}{如果你不喜欢这件商品,可以在三日之内退货。}

    \item sollen(sollten):表达说话人不能确定的条件因素。如果要表现难以预测或实现的条件,需要使用第二虚拟式 sollten 的形式
    
    \example{Falls er die Arbeit nicht allein schaffen sollte, werde ich ihm helfen.}{如果他一个人完不成工作,我就帮他。}

    \example{Sollte er schon gegangen sein, so hinterlasse ihm eine Nachricht.}{如果你已经走了,你就给他留个信。}

    \item 非现实条件句: 由 wenn 引导,主从句谓语进行第二虚拟式的变位
    
    \example{Wenn du Lust hättest, könnten wir Ski fahren.}{如果你有兴趣的话,我们可以去滑雪。}

    \example{Wenn Pauline Zeit gehabt hätte, wäre sie zu uns gekommen.}{要是 Pauline 有时间的话,她就到我们这来了。}

    \example{Hätte sie sich richtig zugedeckt, wäre sie jetzt nicht erkältet.}{她要是盖好被子的话,现在就不会感冒了。}

    \item dass: 与主句的介词词组(unter der Bedingung, unter der Voraussetzung, in dem Falle, \ldots)或者过去分词(vorausgesetzt, angenommen, \ldots),如果是和过去分词连用,主句往往带有 so 或 dann
    
    \example{Im Fall, dass ich das gesamte Erbe meiner Tante bekomme, so höre ich auf zu arbeiten.}{如果我继承了我姑姑的全部遗产,我就不用工作了。}

    \example{Vorausgesetzt, dass du dich beeilst, so erreichst du den Zug.}{你要是抓紧的话,就可以赶得上火车了。}
\end{itemize}

\subsubsection*{让步从句}
描述从句的动作对与主句相反,但从句动作的结果没有起到作用

\begin{itemize}
    \item obwohl, obgleich: 意为「尽管,虽然」,主句常带有 trotzdem, dennnoch, noch 等关联词。口语中常用 wo 来代替,从句一般位于主句后,并带有强调语气的 doch, erst, gerade, schon 等
    
    \example{Obwohl er steinreich ist, wohnt er in einer bescheidenen Mietswohnung.}{虽然他非常有钱,却住在简陋的租来的房子。}

    \example{Die Waren wurden nicht geliefert, obgleich er sie schon bezahlt hatte.}{虽然已经付过钱,但商品还没寄出去。}

    \example{Ich konnte nicht sclafen, obwohl / wo ich ein Sclafmittel genommen habe.}{尽管我服用了安眠药,但还是睡不着。}

    \item obzwar, obschon, wiewohl, trotzdem: 意义与 obwohl, obgleich 基本相同,其中 trotzd\uline{e}m(重音与副词tr\uline{o}tzdem不同)多用于口语,其余常用于书面语
    
    \example{Wiewohl er anfangs überhaupt kein Geld hatte, so brachte er es durch seine kaufmännische Geschicklichkeit zu einem großen Vermögen.}{尽管一开始没有任何资金,但通过他熟练的商业头脑获得大量财富。}

    \example{Er hat das Abitur nicht geschafft, trotzd\uline{e}m er Tag und Nacht gelernt hat.}{他没通过毕业考试,尽管他白天黑夜的学习。}

    \item wenn\ldots auch, auch wenn selbst wenn, sogar wenn: 意为「即使\ldots 就算\ldots 」,主句常带有关联词 doch, immerhin, trotzdem, dennoch, noch 等

    \example{Wenn es auch schon spät ist, wollen wir die Arbeit fortsetzen.}{即使已经很晚了,我们也要继续工作。}

    \example{Selbst wenn wir langsam laufen, erreichen wir den Bus.}{就算我们慢悠悠地走过去,也能赶上公交车。}

    \item nicht einmal wenn: 「就算\ldots 也不」
    
    \example{Die anderen rechnen nie so schnell wie er, nicht einmal wenn sie einen Taschenerchner benutzen.}{就算别人用计算器,也没有他算得那么快。}
    
    \item wenngleich: 「虽然」,常用于科技文献、正式公文、法律语言中
    
    \example{Wenngleich in den letzten Jahren immer mehr Ehen zwischen Partnern verchiederner Nationalität geschlossen wurden, nimmt doch auch die Zahl der Scheidungen zu.}{虽然在过去几年中跨国婚姻越来越多,但离婚的数字也在增长。}

    \item auch: 从句中带有 auch,可以不用连接词来构成。主句常有关联词 doch, dennoch, trotzdem 等。此时从句只能位于主句前。
    
    \example{Benötigen es die Schulen auch so dringend, so wurde das Hallenbad trotzdem / dennoch / doch nicht gebaut.}{尽管学校急需,游泳池依然没有建造起来。}

    \example{Ist es auch sehr kalt, ich fahre trotzdem / dennoch / doch mit dem Rad zur Arbeit.}{尽管很冷,我依然骑自行车去上班。}

    \item W- + auch (immer), (so+形容词/副词) + auch, ob\ldots oder: 加 immer 表示强调,意为「不管\ldots 」,从句位于主句前,主句变位动词在 LSK
    
    \example{Wo sie auch immer nach ihrer Brille suchte, sie fand sie nicht.}{不管在哪里,她都找不到眼镜。}

    \example{Wofür du dich in dieser Frage auch (immer) entscheidest, wir respektieren deinen Entschluss.}{不管在这个问题上你做了怎样的选择,我们都尊重你的决定。}

    \example{Der Politiker wurde nicht gewählt, so überzeugt er auch von sich selbst war.}{这个政治家没有获得选举胜利,不管对自己多么有自信。}

    \example{Ob sie beschäftigt ist oder nicht, sie hilft mir immer.}{不管他忙还是不忙,她总会帮助我。}

    \example{Wir müssen ihn einladen, ob wir ihn mögen oder nicht.}{我们必须邀请他,不管我们是否喜欢他。}
\end{itemize}

\subsubsection*{结果从句}

结果从句用来说明主句动作产生的结果,一般位于主句后。

\begin{itemize}
    \item so dass, sodass: 完全一样

    \example{Er spricht stockend, sodass / so dass man ihn schecht versteht.}{他结结巴巴地讲话,以至于人们很难听懂。}

    \item dass: 强调主句动作产生的影响和效果,通过主句的「关联词+名词/ 形容词」体现,其中形容词带有关联词 so, derart, dermaßen, genug, \ldots,名词带有关联词 so, solch, derartig, \ldots。so, solch 可放在不定冠词前或后,其中 solch 放在冠词后需要变格(包括词尾),derartig 只能放在不定冠词后。

    \example{Der Fahrer ging so schnell in die Kurve, dass er die Kontrolle über sein Fahrzeug verlor.}{司机拐弯的时候行使得非常快以至于对车子失去了控制。}

    \example{Ich hatte [(einen) so / einen solchen / solch einen / einen derartigen / derarigen] großen Hunger, dass ich sofort was essen musste.}{我如此饥饿,必须马上吃东西。}
    
    \item ohne dass, ohne zu: 表示没有实现主句的行为没有实现期望的结果,从句总是位于主句后,常加入 aber, jedoch 等小品词
    
    \example{Die beiden Fahrzeuge rasten aufeinander zu, ohne dass es jedoch zum Zusammenstoß kam.}{两辆车迎面迅速行驶,却没有撞到一起。}

    \example{Wie spielen Lotto, ohne jemals etwas zu gewinnen.}{我们买彩票却从未中奖过。}

    \item (als) dass: 同样表示未实现期望结果,本身具有否定含义,翻译为「太/过于\ldots 以至于不能\ldots」,主句中必须含有 zu, nicht genug, zu wenig, nicht so 等关联词。此时从句动词使用第二虚拟式变位,常用到情态动词 können,并且总是位于主句后
    
    \example{Das Wasser ist zu kalt, als dass man baden könnte.}{水太凉了,人们不能洗澡。}

    \example{Es gab hier nicht genug / zu wenig Stühle, (als) dass alle hätten sitzen können.}{这里没有足够的椅子,以至于不是所有人都有位子坐。}

    \item um\ldots zu: 主从句主语一致可使用此结构,主句必须带有 genug, zu, nicht genug, zu wenig, nicht so 等关联词
    
    \example{Das Problem ist zu wenig wichtig, als dass man sich darüber ärgern sollte.\\
    $\hookrightarrow$Das Problem ist zu wenig wichtig, um sich darüber zu ärgern.}{这个问题没有那么重要,不值得为它生气。}
\end{itemize}

\subsubsection*{目的从句}

目的从句说明主句动作的目的、意图以及打算。

\begin{itemize}
    \item damit, dass, auf dass: 主句可出现 darum, dafür, dazu, deshalb, deswegen, zu dem Zweck, in der Absicht 等关联词。此时 dass 意义与 damit 相同,但 dass 主要用于口语中,而为了与其他从句区分,从句中变位动词常是第一虚拟式。

    \example{Wir bleiben in der Nähe der Reiseleiterin, damit wir ihre Erklärung gut verstehen können.}{我们站在导游身边,为了能听清楚她的讲解。}

    \example{Wir haben die Wohnung deshalb genau vermessen, dass die Möbel später auch hineinpassen.}{我们仔细丈量了房间,以便日后合适的添置家具。}

    \example{Sie schließen die Tür ab, dass keiner hereinkomme.}{他们锁上门,为的是不让任何人进来。}

    \item um\ldots zu: 主从句主语必须一致(非主句宾语与从句主语一致)

    \example{Er benutzt die Lupe, damit er die Schrift liest.\\
    $\hookrightarrow$Er benutzt die Lupe, um die Schrift zu lesen.}{他使用放大镜,以便看清字迹。}

    \example{Vom Notarzt wurde alles getan, um das Leben des Verunglückten zu retten.}{急救医生尽最大力抢救不幸者的生命。}
\end{itemize}

\subsubsection*{对比从句}

从句通过对比突出主句的动作行为,从句位置不限,主要连接词为 während,也可以使用 dagegen, wohingegen, indes, indessen。

\example{Ich arbeite wie verrückt, während du faul auf dem Sofa liegst.}{我在疯狂地工作,而你却懒散地躺在沙发上。}

\example{Während es gestern noch geregnet hat, scheint heute die Sonne.}{昨天还下着雨,今天就太阳高照了。}

\example{Er ist blond, wohingegen seine Schwester beinahe schwarzhaarig ist.}{他满头金黄头发,而他妹妹却几乎是黑发。}

\subsubsection{定语从句}

定语从句(der Attributivsatz)修饰名词,代词或整个句子。包括由关系代词、关系副词、代副词引导的关系从句(der Relativsatz)以及其他情况引导的从句。

\subsubsection*{定语从句的位置}
定语从句的位置由指代成分决定,通常情况下直接位于指代成分的后面,但是也可以不紧跟。如果指代成分是一个比较长的词组,则可以跟在整个词组的后面;如果主句动词与被指代成分关系非常紧密且从句结构复杂,从句可以位于 NF。

\example{Die Süßigkeiten, die die Großmutter mitgebracht hat, liegen auf dem Tisch.}{祖母带来的甜食放在桌子上。}

\example{Er empfing \uline{die Gäste aus der Schweiz}, die am Symposium teilnahmen.}{他接待了来参加研讨会的瑞士客人。}

\example{Herr Meier hat das Auto verkauft, mit dem er nach Afrika gefahren ist.}{Meier 先生卖掉了他曾经开着去过非洲的汽车。}

\subsubsection*{由关系代词引导}
\begin{itemize}
    \item 由关系代词 der, welch引导: 关系代词 der 的变格与指示代词 der 一致,参考 \cref{tab:demonstrativpronomen-declensionn}与\cref{sec:relativnoun}。同时关系代词的 G 个可以加形容词修饰名词。

    \example{Der Mann, \uline{der} dort steht, liest eine Zeitung.}{站在那边的那个男人在读报纸。}

    \example{Die nungernden, \uline{denen} man zahlreiche Hilfspakete geschickt hatte, waren sehr dankbar.}{收到大量救灾物资的饥饿的人们非常感激。}

    \example{Der Junge, dessen Vater in einer anderen Stadt arbeitet, wohnt bei seinen Großeltern.}{那个孩子的父亲在另一个城市工作,他住在他祖父母那。}

    \example{das Vorgehen, das das Komitee vorsclägt\\
    $\hookrightarrow$das Vorgehen, welches das Komitee vorschlägt}{委员会建议的工作方法}

    \item 介词+关系代词引导: 如果主句中被指代成分是名词或代词,而连接词在从句中充当介词宾语或有介词的状语,则关系代词位于介词之后
    
    \example{Die Adresse des Redakteurs, \uline{an den} sie sich wandte, hatte sie im Telefonbuch gefunden.}{她要去找的那个编辑的地址,她在电话本中找到了。}

    \example{Der Junge, \uline{mit dessen} Eltern wir befreundet sind, spielt oft bei uns im Garten.}{那个小孩经常在我们的院子玩耍,我们跟他的父母是好朋友。}

    \example{Das Gebäude, in dem ich arbeite, hat eine Klimaanlage.\\
    Das Gebäude, worin ich arbeite, hat keine Klimaanlage.}{我工作的大楼内没有空调。}

    \item was 引导: 主句的被指代成分包括不定代词 / 不定数词 / [das+形容词最高级作名词],或 was 修饰整个句子。但是,如果形容词最高级后修饰名词,依然要用 der / welch 修饰,如果被指代成分为 A 或 G 格,使用 wessen / dessen 和 dem

    \example{Es gibt vieles, was mir nicht klar ist.}{有很多地方我不清楚。}

    \example{Der Unfall ist schon \uline{das Schlimmste}, was überhaupt passieren kann.}{这起事故已经是能发生的最糟糕的了。}

    \example{Wir sind ins Kino gegange, was wir schon lange nicht mehr gemacht hatten.}{我们去看了电影,我们好久都没去看过电影了。}

    \example{Das ist \uline{das interessanteste Buch}, das ich gelesen habe.}{这是我读过的最有趣的一本书。}

    \example{Es gibt nichts, dem er aus dem Weg gehen würde.}{没有什么他不敢做的事情。}

    \item wer 引导: 在从句中之人,从句在主句前,同时主句中往往带有指示代词 der。同时如果主句和从句的代词的格相同,指示代词可以省略,不一致则不可以

    \example{Wem der Kuchen nicht schmeckt, der kann was anders zugreifen.}{谁不爱吃这个糕点可以吃些其他的。}

    \example{Wer damit verstanden ist, kann nach Hause gehen.}{谁同意谁就可以回家。}
\end{itemize}

\subsubsection*{由代副词引导}
如果从句的动词还要支配介词,并且符合以下条件,则可以使用代副词 wo(r)+介词进行引导。

\begin{itemize}
    \item 不定代词和不定数词作名词。此种情况,在口语中,也可以用「介词+das/dem」代替代副词

    \example{Eine Weltereise ist \uline{etwas}, \uline{von} dem wir schon lange träumen.\\
    $\hookrightarrow$Eine Weltereise ist etwas, \uline{wovon} wir schon lange träumen.}{环球旅行是我一直以来的梦想。}

    \item das+形容词最高级作名词

    \example{Geld war nicht das Einzige, \uline{mit} dem er sie untersützte.\\
    $\hookrightarrow$Geld war nicht das Einzige, \uline{womit} er sie untersützte.}{前并不是他给她提供的唯一支持。}

    \item 代指整个句子

    \example{Er hat mich eingeladen, worüber ich mich sehr freute.}{他邀请了我,我对此感到非常高兴。}
\end{itemize}

\subsubsection*{由关系副词引导}

\begin{itemize}
    \item 如果主句被指代成分为表达地点的名词或副词,并且在从句中作为地点或方向说明语,由 wo / woher / wohin 引导,如果名词带有冠词,也可以用「介词+der」的形式。

    \example{Sie fahren in das Skigebiet, wohin/in das auch wir fahren.}{他们去滑雪区,我们也要去那儿。}

    \item 也可以用 wo, wenn 和 als 表示时间,其中 wenn 用于现在/将来时态,als 用于过去时态
    
    \example{Erinnerst du dich noch an den Tag, wo wir uns zum ersten Mal gesehen haben?}{你还能记得我们第一次见面的那天吗?}

    \example{Wir möchten dich jetzt, wo/wenn du so beschäftigt bist, nicht mit Fragen belästigen.}{你现在这么忙,我们不想在这个时间问你问题打扰你。}

    \example{Das was die Zeit, wo/als geheiratet hat.}{那就是他结婚的时间。}
\end{itemize}

\subsubsection*{其他形式引导}
除了上述的关系词,以下结构也可以引导定语从句。

\begin{itemize}
    \item 分词结构

    \example{Eine Medienkampagne, \uline{gerichtet} auf Jugendliche, soll den Alkoholmissbrauch eindämmen.}{一场针对于青少年的媒体运动应该致力于阻止酗酒现象。}

    \item dass 引导,通常指向名词化的动词或形容词,同时可以改写为主语或宾语从句

    \example{Es besteht die Hoffnung, dass es noch Überlebende gibt.\\
    $\hookrightarrow$Man hofft, dass es noch Überlebende gibt.}{人们希望还会有生还者。}

    \example{Die Wahrscheinlichkeit, dass du Recht hast, ist gering.\\
    $\hookrightarrow$Dass du Recht hast, ist kaum wahrscheinlich.}{你基本上是没有道理的。}

    \item 不定式结构

    \example{Die Schwierigkeit, eine Wohnung zu finden, wird immer größer.\\
    $\hookrightarrow$Eine Wohnung zu finden wird immer schwieriger.}{找到一间房子是越来越困难了。}

    \item 无连接词引导

    \example{Ich bin der Meinung, man soll noch mehr Maßnahmen gegen Umweltverschmutzung ergreifen.\\
    $\hookrightarrow$Ich bin der Meinung, dass man noch mehr Maßnahmen gegen Umweltverschmutzung ergreifen soll.}{我认为人们还要采取更多的措施来防治环境污染。}

    \item ob 或 W- 引导的从句

    \example{Er stellte die Frage, ob / wann er kommen dürfe.}{他问道他[是否/何时]可以过来。}
\end{itemize}

\subsubsection{表语从句}

\begin{itemize}
    \item 较少见,一般由 dass, was 或 wie 引导,即「是\ldots ,正如\ldots 一样」

    \example{Die Frage ist, wie wir ihn dazu überzeugen können.}{问题是,我们怎样才能说服他。}
    
    \example{Wir finden die Wohnung, wie wir sie schon immer gefunden haben (nämlich schlecht)}{这间房和我们一直以来想象的一样(差)。}
\end{itemize}

\subsection{特殊从句}

\subsubsection{嵌入式从句}

\begin{itemize}
    \item 嵌入式从句(der Schaltsatz) 是指从句与主句由破折号隔开,从句不带有连接词,虽然从形式上与主句并列,但是意义上受主句支配。如果要在口语中使用,叙述时应保留一定的停顿,书面语需要用破折号隔开

    \example{Der klein Josef, der damals erst fünf Jahre alt war, spielte einen Walzer.\\
    $\hookrightarrow$Der klein Josef - er war damals erst fünf Jahre als - spielte einen Walzer.}{小 Josef —— 他那是还只有五岁 —— 跳起了华尔兹。}

    \example{Du musst, wie ich dir schon gestern gesagt habe, deine Eltern anrufen.\\
    $\hookrightarrow$Du musst - ich habe es dir gestern schon gesagt - deine Eltern anrufen.}{你必须 —— 我昨天就跟你说过 —— 给你父母打电话。}
\end{itemize}

\subsubsection{从句嵌套}

由于从句可以替代句子成分,因此从句嵌套从句有时可达一段,也成为句段(die Satzperiode),这种情况通常只在文学或科学作品中出现。如:

Da er daheim seine Zeit vertat, beim Unterricht langsamen und abgewandten Geistes war und bei den Lehrern schlecht angeschrieben stand, so brachte er beständig die erbärmlichsten Zensuren nach Hause, worüber sein Vater, ein langer, sorgfältig gekleideter Herr mit sinnenden blauen Augen, der immer eine Feld blume im Kopfloch trug, sich sehr erzürnt und bekümmert zeigte.

{\color{codegray}因为他在家里虚度时光,在课堂上拖拖拉拉心不在焉,给老师留下的印象很坏,所以带回家的成绩单总是很糟。她的父亲,身材高大穿着讲究,有着蓝色深邃的眼睛,总是在耳边插一朵野花,对此很是恼火、焦虑。}

\subsubsection{从句减缩}

在句子中,如果一个句子成分先后出现两次,则第二次出现时省略的情况称为从句减缩(zusammengezogener Satz)。\textbf{很多情况下,可以减缩:}
\begin{itemize}
    \item 相同主语

    \example{\uline{Das Buch} liegt in der Bibliothek und \uline{das Buch} kann dort eingesehen werden.\\
    $\hookrightarrow$\uline{Das Buch} ligt in der Bibliothek und kann dort eingesehen werden.}{这本书在图书馆,可以到那里阅读。}

    \item 相同宾语

    \example{Ich buche \uline{die Reise} und du bezahlst \uline{die Reise}.\\
    $\hookrightarrow$Ich buche und bezahlst \uline{die Reise}.}{这次旅行我来预订,你来结帐。}

    \item 相同状语

    \example{Sie fuhren \uline{nach Paris} und wird flogen \uline{nach Paris}.\\
    $\hookrightarrow$Sie fuhren und wir flogen \uline{nach Paris}.}{它们乘车去巴黎,我们乘飞机去。}

    \item 相同动词、助动词、情态动词

    \example{Er \uline{bestellte} einen Wein. Sie \uline{bestellte} ein Bier.\\
    $\hookrightarrow$Er \uline{bestellte} einen Wein, sie ein Bier.}{他点了一杯红酒,她点了啤酒。}

    \example{Sie \uline{wurden} befördert und wir \uline{wurden} entlassen.\\
    $\hookrightarrow$Sie \uline{wurden} befördert und wir entlassen.}{他们被升职了,我们被解雇了。}

    \example{Der Lehrer \uline{soll} erklären und der Schüler \uline{soll} zuhören.\\
    $\hookrightarrow$Der Lehrer \uline{soll} erklären und der Schüler zuhören.}{老师应该讲解,学生应该听讲。}

    \item 相同主语与谓语

    \example{\uline{Er hat} seiner Schwester ein Buch \uline{geschenkt} und \uline{er hat} seinem Bruder eine CD \uline{geschenkt.}\\
    $\hookrightarrow$\uline{Er hat} seiner Schwester ein buch und seinem Bruder eine CD \uline{geschenkt}.}{他送给他妹妹一本书,他送给他弟弟一张CD。}

    \example{\uline{Ich muss} etwas einkaufen und \uline{ich muss} dann schnell nach Hause gehen.\\
    $\hookrightarrow$\uline{Ich muss} etwas einkaufen und dann schnell nach Hause gehen.}{我得去买点东西,然后赶紧回家。}
\end{itemize}

\textbf{但在以下这些情况,不能减缩:}
\begin{itemize}
    \item 代词相同但格不同

    \example{Er hat \ult{uns}{D} eine Karte geschickt und \ult{uns}{A} zum Essen eingeladen.}{他给我们寄来一张卡片,并请我们吃饭。}

    \item 对 Sie 的命令式中,Sie不能省略

    \example{Treten \uline{Sie} ein und nehmen \uline{Sie} Platz.}{您请进,请坐。}

    \item 动词的结构不同,不能省略

    \example{Er \uline{fürchtete sich vor} der Einsamkeit und der Dunkelheit.}{他怕孤单,而且怕黑。}

    \item 动词形式相同但意义不同

    \example{Wir \uline{laden} zuerst das Gepäck in den Kofferraum ein und \uline{laden} euch dann zum kaffee ein.}{我们先把行李放进行李箱然后请你们去喝咖啡。}

    \item 动词形式相同但功能不同

    \example{Sie \ult{hat}{实义动词} ein Auto und \ult{hat}{助动词} uns schon oft damit abgeholt.}{她有车,经常开车接我们。}

    \example{Er \ult{nahm}{实义动词} einen Hammer und \ult{nahm}{功能动词} damit die Arbeit in Angriff.}{他拿起锤子开始了工作。}

    \example{Er \ult{lebte}{实义动词} in München und dort \ult{lebte er auf großen Fuß.}{固定搭配}}{他在慕尼黑过着大手大脚的生活。}

\end{itemize}