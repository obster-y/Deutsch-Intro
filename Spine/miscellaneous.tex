\section{其他}
\subsection{副词}
德语没有副词化形容词的词尾「-ly」,只要意义符合,形容词可以直接当作副词使用。同时「-erweise」类似于「-ly」,不过是修饰整个从句而非单个动词或形容词。


\begin{table}[htbp]
    \centering
\begin{tblr}{
    width=\linewidth,
    colspec={X[l,1.5]X[l,3.5]||X[l,1.5]X[l,3.5]},
}
    \SetCell[c=2]{c} 地点副词 &                 & \SetCell[c=2]{c} 疑问副词 & \\
    \hline
    aufwärts           & upward                 & warum         & why         \\
    auseinander        & separated, apart       & was           & what        \\
    außen              & outside                & wann          & when        \\
    da                 & here, there            & wie           & how         \\
    drüben             & over there             & wo            & where at    \\
    geradeaus          & straight ahead         & woher         & where from  \\
    hier               & here                   & wohin         & where to    \\
    hinten             & in back, at the back   &               &             \\
    innen              & inside                 &               &             \\
    links              & left                   &               &             \\
    nebenan            & next door              &               &             \\
    nirgend(s/wo)      & nowhere                &               &             \\
    oben               & above                  &               &             \\
    rechts             & right (not left)       &               &             \\
    rückwärts          & backwards              &               &             \\
    überall            & everywhere             &               &             \\
    unten              & under                  &               &             \\
    vorn(e)            & in front, at the front &               &             \\
    vorwärts           & forward                &               &             \\
\end{tblr}
\end{table}

\begin{table}[htbp]
    \centering
\begin{tblr}{
    width=\textwidth,
    colspec={X[l,1.5]X[l,3.5]||X[l,1.5]X[l,3.5]},
}
    \SetCell[c=2]{c} 时间副词 &             & \SetCell[c=2]{c} 行为副词 &               \\
    \hline
    anfangs    & at the beginning           & beinahe        & nearly, almost           \\
    bald       & soon                       & besonders      & especially               \\
    bereits    & already                    & bloß           & merely, simply, just     \\
    damals     & at that time, then         & daneben        & besides, in addition     \\
    danach     & afterward                  & ebenfalls      & likewise, also           \\
    dann       & then                       & ebenso         & equally, similarly       \\
    diesmal    & this time                  & eigentlich     & actually, in fact        \\
    einmal     & once                       & fast           & almost                   \\
    endlich    & finally, at last           & gemeinsam      & in common, jointly       \\
    früher     & earlier, previously        & gern, gerne    & gladly                   \\
    gestern    & yesterday                  & hoffentlich    & hopefully                \\
    heute      & today                      & insgesamt      & in total, altogether     \\
    immer      & always                     & kaum           & hardly                   \\
    inzwischen & meanwhile                  & leider         & unfortunately            \\
    jemals     & ever                       & mindestens     & at least, at minimum     \\
    jetzt      & now                        & nämlich        & namely                   \\
    langsam    & slowly                     & natürlich      & naturally                \\
    längst     & long ago                   & nebenbei       & by the way, incidentally \\
    manchmal   & sometimes                  & schließlich    & finally                  \\
    meistens   & mostly, most often         & sehr           & very                     \\
    neulich    & recently                   & sogar          & even                     \\
    nie        & never                      & sonst          & otherwise                \\
    noch       & still                      & teilweise      & partially                \\
    nochmal    & again                      & übrigens       & by the way               \\
    nun        & now                        & ungefähr       & approximately            \\
    oft        & often                      & ursprünglich   & originally               \\
    schon      & already                    & vielleicht     & perhaps, maybe           \\
    sofort     & immediately                & wahrscheinlich & probably, likely         \\
    vorbei     & over, past                 & wirklich       & really, truly            \\
    vorher     & previously, before         & ziemlich       & rather, quite            \\
    vorhin     & just now, a short time ago & zufällig       & accidentally, by chance  \\
    wieder     & again                      & zurück         & back                     \\
    zuerst     & first, at first            & zusammen       & together                 \\
    zurzeit    & at the moment, at present  &                &                          \\
\end{tblr}
\end{table}

\clearpage
\subsection{连词}
德语对于并列连词与从属连词分得比较清楚。

\begin{longtblr}[
    theme=nocaption,
    presep={12pt},
]{
    width=\linewidth,
    rowhead=1,
    colspec={cX},
    columns={l,colsep=3pt},
    rows={m},
}
\textbf{Conjunction} & \textbf{Sample usage and remark} \\
    \hline
\SetCell[c=2]{l}{\textbf{Subordinating Conjunctions}} \\
    \hline
    \SetCell[r=6]{c,m}{dass\\ (that)} & Ich will, dass du deine Hausaufgaben machst. \\
    & I want you to do your homework. \\
    & (literally I want that you do your homework) \\
    \cline{2-2}
    & Er sagte, dass er kommen würde. \\
    & He said (that) he's coming. \\
    & \textcolor{codegray}{[notice that "dass" is often required in German where "that" is optional in English.]} \\
    \hline
    \SetCell[r=5]{c,m}{wenn\\ (if, when)} & Wenn du möchtest, kannst du bei mir bleiben. \\
    & If you'd like, you can stay with me. \\
    \cline{2-2}
    & Jeden Tag, wenn ich von der Schule nach Hause kam, wartete mein Hund schon vor der Tür. \\
    & Every day, when I came home from school, my dog was already waiting at the door. \\
    \cline{2-2}
    & \textcolor{codegray}{"When" is a more differentiated concept in German. "Wenn" is used for the present, future, and repeated events in the past ("Remember when we used to stay out all night long?"). For a single event in the past ("When I got there, the house was empty") you should use "als".} \\
    \hline
    \SetCell[r=2]{c,m}{immer wenn\\ (whenever)} & Immer wenn es regnet, muss ich an dich denken. \\
    & Whenever it rains, I think of you. \\
    \hline
    \SetCell[r=3]{c,m}{weil\\ (because)} & Ich komme zu spät, weil ich eine Reifenpanne hatte. \\
    & I'm arriving late because I had a flat tire. \\
    \cline{2-2}
    & \textcolor{codegray}{Many Germans also use weil in speech as a coordinating conjunction, without moving the verb to the end afterwards. Often they're not even aware of it, although they would instantly see the mistake if it was used this way in writing. One possibility is that they're doing it for sentences where the listener already knows the "reason" (as with denn above). Another is that they're accommodating a longer subordinate clause that would be harder to understand with the verb at the end.} \\
    \hline
    \SetCell[r=2]{c,m}{sobald\\ (as soon as)} & Wir reden darüber, sobald ich zurück bin. \\
    & We'll talk about it as soon as I'm back. \\
    \hline
    \SetCell[r=2]{c,m}{bevor\\ (before)} & Ich hoffe, wir sehen uns noch mal, bevor ich das Land verlasse. \\
    & I hope we see each other again before I leave the country. \\
    \hline
    \SetCell[r=4]{c,m}{nachdem\\ (after)} & Sie kamen an, nachdem alles schon vorbei war. \\*
    & They arrived after everything was already over. \\*
    \cline{2-2}
    & Nachdem alles vorbei war, konnten wir darüber reden. \\*
    & After it was all over, we were able to talk about it. \\*
    \hline
    \SetCell[r=3]{c,m}{als\\ (as, when, while)} & Ich bin gestolpert, als ich aus dem Bus stieg. \\
    & I stumbled as I was getting off the bus. \\
    \cline{2-2}
    & \textcolor{codegray}{[Again, als is for a single past event; use wenn for a repeated event.]} \\
    \hline
    \SetCell[r=2]{c,m}{da\\ (because, since)} & Sie fuhren gerne Rad, da ihnen die Bewegung gut tat. \\
    & They enjoyed cycling because being active did them good. \\
    \hline
    \SetCell[r=2]{c,m}{seit, seitdem\\ (since \\~ [point in time])} & Er konnte an nichts anderes mehr denken, seit[dem] er sie zum ersten Mal gesehen hatte. \\
    & He couldn't think of anything else since he had seen her for the first time. \\
    \hline
    \SetCell[r=3]{c,m}{ob\\ (whether)} & Ich weiss nicht, ob er alt genug ist. \\
    & I don't know whether he's old enough. \\
    \cline{2-2}
    & \textcolor{codegray}{Often translates to "if" in casual English, "I don't know if he's old enough"} \\
    \hline
    \SetCell[r=2]{c,m}{obwohl\\ (although)} & Ich habe zugesagt, obwohl ich Bedenken hatte. \\
    & I agreed, although I had my doubts. \\
    \hline
    \SetCell[r=2]{c,m}{solange\\ (so long as)} & Solange du hier bist, kannst du dich nützlich machen. \\
    & As long as you're here, you can make yourself useful. \\
    \hline
    \SetCell[r=2]{c,m}{während\\ (while, during)} & Während du auf mich wartest, kannst du die Zeitung lesen. \\
    & While you're waiting for me, you can read the paper. \\
    \hline
    \SetCell[r=2]{c,m}{bis\\ (until)} & Ich bleibe hier, bis du zurückkommst. \\
    & I'll wait here till you come back. \\
    \hline
\SetCell[c=2]{l}{\textbf{Coordinating Conjunctions}} \\
    \hline
    \SetCell[r=2]{c,m}{und\\ (and)} & Du machst das noch mal und ich gehe ohne dich. \\
    & You do that again and I'm going without you. \\
    \hline
    \SetCell[r=2]{c,m}{oder\\ (or)} & Du hörst auf damit oder ich gehe. \\
    & You stop it or I'll leave. \\
    \hline
    \SetCell[r=2]{c,m}{aber\\ (but)} & Ich wollte es abholen, aber der Laden hatte schon zu. \\
    & I wanted to pick it up, but the store was already closed. \\
    \hline
    \SetCell[r=3]{c,m}{doch\\ (but, and yet)} & Jahre vergehen, doch die Liebe bleibt. \\
    & Years go by, but love remains. \\
    \cline{2-2}
    & \textcolor{codegray}{[more lyrical and nuanced than aber, often expresses a mild paradox (it's like this, and yet it's also like that)]} \\
    \hline
    \SetCell[r=3]{c,m}{denn\\ (because, for)} & Bleibe bei uns, denn es will Abend werden. \\*
    & Stay with us, for evening is nigh. \\*
    \cline{2-2}
    & \textcolor{codegray}{a bit softer than "weil" and often used when the hearer is already aware of the following explanation.} \\*
    \hline
\SetCell[c=2]{l}{\textbf{Two-Part Conjunctions}} \\
    \hline
    \SetCell[r=2]{c,m}{entweder…order\\ (either… or)} & Wir können entweder ins Kino oder ins Restaurant gehen. \\
    & We can either go to the cinema or to the restaurant. \\
    \hline
    \SetCell[r=2]{c,m}{weder…noch\\ (neither… nor)} & Sie isst weder Fisch noch Fleisch. \\
    & She eats neither fish nor meat. \\
    \hline
    \SetCell[r=2]{c,m}{sowohl. als auch\\ (both… and)} & Dieses Gerät spart sowohl Zeit als auch Geld. \\
    & This appliance saves both time and money. \\
    % & This appliance saves time as well as money. \\
    \hline
    \SetCell[r=2]{c,m}{sowohl…wie\\ (as well as)} & sowohl im Inland wie auch im Ausland \\
    % & both at home and abroad \\
    & at home as well as abroad. \\
    \hline
    \SetCell[r=2]{c,m}{Je…desto\\ (more… more…)} & Je mehr du redest, desto dümmer klingst du. \\
    & The more you talk, the dumber you sound. \\
    \hline
    \SetCell[r=3]{c,m}{Zwar… aber…\\ (it's true… but…)} & Er mochte sie zwar, wollte sie aber nicht heiraten. \\
    & While he certainly liked her, he did not want to marry her. \\
    \cline{2-2}
    & \textcolor{codegray}{indeed / granted}\\
\end{longtblr}

\clearpage
\subsection{语气助词}

\begin{longtblr}[
    theme=nocaption,
    presep={0pt},
]{
    width=\linewidth,
    rowhead=1,
    colspec={clX[l]},
    columns={l,colsep=3pt},
    rows={m},
}
    \SetCell[c=2]{l} \textbf{Modal Particles} & & \textbf{Sample usage} \\
    \hline
    \SetCell[r=12]{l,m}{Affirmation /\\ Agreement} & \SetCell[r=4]{l,m}{aber} & aber gerne! \\
    &       & with pleasure! \\
    \hline
    &       & aber sicher! \\
    &       & most certainly \\
    \hline
    & \SetCell[r=2]{l,m}{wohl} & Das ist wohl wahr! \\
    &       & That's certainly true! \\
    \hline
    & \SetCell[r=2]{l,m}{ja} & Das ist ja eine tolle idee! \\
    &       & That's really a great idea! \\
    \hline
    & \SetCell[r=4]{l,m}{na} & Na klar komme ich! \\
    &       & You bet I'm coming! \\
    \hline
    &       & Na logisch! \\
    &       & Of course! \\
    \hline
    \SetCell[r=4]{l,m}{Contradiction /\\ Disagreement} & \SetCell[r=4]{l,m}{doch} & Du bist doch nur zugekifft. \\
    &       & You're just [saying that because you're] high. \\
    \hline
    &       & Q: Das ist doch nicht dein Ernst, oder? A: Doch! \\
    &       & Q: You're not being serious, are you? A: I am! \\
    \hline
    \SetCell[r=6]{l,m}{Emphasis /\\ Focus} & \SetCell[r=4]{l,m}{gerade} & Dass ich das gerade von DIR höre... \\
    &       & That I'm hearing that from YOU (of all people)… \\
    \hline
    &       & Gerade heute musste es schneien! \\
    &       & It had to snow today (of all days)! \\
    \hline
    & \SetCell[r=2]{l,m}{eben} & Ich versuche, eine Antwort auf eben die Frage zu formulieren. \\
    &       & I'm trying to find an answer to [just] that very question. \\
    \hline
    \SetCell[r=6]{l,m}{Resignation} & \SetCell[r=2]{l,m}{eben} & So ist es eben. / Es ist eben so. \\
    &       & That's just how it is. \\
    \hline
    & \SetCell[r=2]{l,m}{naja} & Naja, was hast du erwartet? \\
    &       & Ah well, what did you expect? \\
    \hline
    & \SetCell[r=2]{l,m}{halt} & Ich war halt besoffen. \\
    &       & (What can i say?) I was drunk. \\
    \hline
    \SetCell[r=4]{l,m}{Surprise} & \SetCell[r=2]{l,m}{aber} & Das war aber nett von dir! \\
    &       & That was nice of you! [I wasn't expecting it] \\
    \hline
    & \SetCell[r=2]{l,m}{etwa} & Ist das etwa für mich? \\
    &       & Is that for me? \\
    \hline
    \SetCell[r=4]{l,m}{Interest} & \SetCell[r=2]{l,m}{denn} & Wie alt bist du denn? [to a child] \\*
    &       & So how old are you? \\*
    \hline
    & \SetCell[r=2]{l,m}{mal} & Guck dir das mal an! \\*
    &       & Take a look at that! \\*
    \hline
    \SetCell[r=6]{l,m}{Intensifiers} & \SetCell[r=2]{l,m}{schon} & Das ist schon viel! \\
    &       & It's more than you think/more than it seems \\
    \hline
    & \SetCell[r=2]{l,m}{ja} & Du bist ja blöd! \\
    &       & Are you ever dumb! \\
    \hline
    & \SetCell[r=2]{l,m}{aber} & Das ist aber völliger Quatsch! \\
    &       & That's complete nonsense! \\
    \hline
    \SetCell[r=8]{l,m}{Exasperation /\\ Anger} & \SetCell[r=4]{l,m}{nur} & Wie konntest du nur? \\
    &       & How COULD you? \\
    \hline
    &       & Was hat er sich nur dabei gedacht? \\
    &       & What WAS he thinking? \\
    \hline
    & \SetCell[r=2]{l,m}{schon} & Was will er schon von mir? \\
    &       & What in the world does he want from me? \\
    \hline
    & \SetCell[r=2]{l,m}{nun} & Was soll das nun bedeuten? \\
    &       & Now what's that supposed to mean? \\
    \hline
    \SetCell[r=4]{l,m}{Softening /\\ Casual} & \SetCell[r=2]{l,m}{halt} & Es war halt ein Vorschlag. \\
    &       & It was just a suggestion. \\
    \hline
    & \SetCell[r=2]{l,m}{mal} & Warte mal. \\
    &       & Wait a sec.
\end{longtblr}

