\section{静词与变格}

\subsection{格与基础变格表}

名词是语言中的重要组成部分,而冠词、形容词、数词修饰名词,代词可替代名词,它们的形式变化受名词而不受动词的影响。因此将中心词(名词)与限定词(冠词、形容词、数词等)统称为一种与动词相对的词类,这里称为「静词」。

语法格(case/Fall),也称格,是一种词法学上的概念,是一种为了反映静词所起语法功能(主语、直接宾语、间接宾语、方位等)的系统,可以正式的定义为“一种根据独立名词成分与中心词的关系类型,标记这些名词成分的系统”。

屈折语都有格的变化,德语具有四种格:主格(Nominativ, N),宾格(Akkusativ, A),与格(Dativ, D),属格(Genitiv, G),分别起到主语、直接宾语、间接宾语、所有格的作用。

因此,由于名词或代词需要承担的不同的语法功能,受到不同语法范畴的影响,它们自身修饰它们的词也要随着变化,这种变化称为变格(Deklination),变格主要体现在词尾的变化。

德语中静词的变格可以总结为以下三张表格,一种基础形式与两种变体。表中的加号表示被修饰的名词也要发生变化(添加词尾)。

base与var1主要应用在冠词、代词的变化上:var1中的黄色是仅有的三处与base的不同之处,因此不需要单独记忆表var1,只要记住三个不同之处即可。

base与var2主要应用在形容词的变化:表var2是形容词的弱变化形式,黄色部分表示形容词混合变化时使用强变化词尾的位置,绿色部分表示形容词强变化时使用弱变化词尾的位置。

\begin{table}[htbp]
    \caption{基础变格表}
    \label{tab:basic-declension}
    \centering
\begin{tblr}{
    width=\textwidth,
    colspec={c|llll|llll|llll},
    cell{1}{2,6,10}={c=4}{c},
    cell{1}{1}={r=2}{c,m},
    cell{3}{6,7,10,11}={markyellow},
    cell{4}{7,11}={markyellow},
    cell{5}{13}={markgreen},
    cell{6}{10,11}={markgreen},
}
    case & base &&&& var1 &&&& var2 \\
    & m.    & n.    & f.    & pl.   & m.    & n.    & f.    & pl.   & m.    & n.    & f.    & pl. \\
    \hline
    N     & er    & es    & e     & e     & --    & --    & e     & e     & e     & e     & e     & en  \\
    A     & en    & es    & e     & e     & en    & --    & e     & e     & en    & e     & e     & en  \\
    D     & em    & em    & er    & en+n  & em    & em    & er    & en+n  & en    & en    & en    & en  \\
    G     & es+s  & es+s  & er    & er    & es+s  & es+s  & er    & er    & en    & en    & en    & en  \\
\end{tblr}
\end{table}

\subsection{名词}

\subsubsection{名词的性和数}
名词通常使用「冠词+主格名词」的形式记录。

不同于英语,除了数(Numerus),名词还具有性(Genus)的区分,即阳性(maskulin)、中性(neutral)、阴性(femini)。为方便记忆,可以将复数当作“第四种性”。名词的性与其现实意义有一定的关系,但更多地是一种语法意义,比如 das Mädchen 的意思是小女孩,但却是中性词,这是由于它是 die Magd (少女) 的“指小词”,所以语法上统一当作中性词处理。因此在记忆名词时,需要将其的性属记住。

总地来说,名词的性有一定的规律可寻,但也不必记住所有名词的性,尤其实在口语交谈中,不太常用的词的性搞错无可厚非,也不影响沟通。因此只需要记住基本词汇的性,以及了解一些规律即可。

在“数”的范畴上,名词具有单数和复数形式,名词的复数形式通常要进行变化,可分为六种(若考虑元音音变,则九种)。

\begin{table}[htbp]
    \caption{名词的复数变化}
    \label{tab:noun-plural}
    \centering
\begin{tblr}{
    width=\textwidth,
    colspec={ccc},
}
    变化形式 & 单数 & 复数 \\
    \hline[0.7pt]
    无词尾变化 & das Messer (knife) & die Messer \\
    \textcolor{codegray}{并有音变} & das Mantel (coat) & die M\e{ä}ntel \\
    \hline
    加-e   & der Schuh (shoe) & die Schuh\e{e} \\
    \textcolor{codegray}{并有音变} & die Wurst (sausage) & die W\e{ü}rst\e{e} \\
    \hline
    加-er  & das Lied (song) & die Lid\e{er} \\
    \textcolor{codegray}{并有音变} & der Wald (forest) & die W\e{ä}ld\e{er} \\
    \hline
    加-(e)n & die Ampel (traffic light) & die Ampel\e{n} \\
    \hline
    加-s   & das Büro (office) & die Büro\e{s} \\
    \hline
    不规则变化 & das Datum (date) & die Daten \\
\end{tblr}
\end{table}

单数有三种性属,再加上复数,可以将复数看作第四种“性”以简化理解/记忆。因此,名词有四种“性”,而每一种性都有四种变格,所以总的来说,一共有十六种变格形式。变格时,除了修饰名词的词的词尾需要按照这些规律变化,有时名词自身也要变化。

\subsubsection{弱变化名词}

有一些{\bf 阳性}名词被称为弱变化名词,因为除了单数主格外,复数及其他格的名词添加「-n」变形。这些名词数量较多,但有规律可寻。

\begin{enumerate}[leftmargin=3.5em, topsep=0pt, itemsep=0pt, parsep=0pt]
    \item 以 -e 结尾且通常是指人/动物的词
    \begin{enumerate}
        \item 阳性名词单数结尾是「-e」但不是常见弱变化名词的只有 der Käse
        \item 但是有些词在复数时不是添加「-n」而是添加「-ns」,这些词多因为其不指人或动物,如 der Wille
        \item 非阳性名词但也遵循此规律的有 das Herz
    \end{enumerate}
    \item 有特定拉丁语/希腊语词尾的词,同时看起来和英语相同意义单词一模一样
    \item 需要单独记忆的特殊名词
\end{enumerate}
\begin{table}[H]
    % \caption{弱变化名词举例}
    \label{tab:n-declension-nouns}
    \centering
\begin{tblr}{
    width=\textwidth,
    cell{odd}{1}={r=2}{c,m},
}
        Type 1 & der Kunde & der Neffe & Der Russe & der Schewede \\
        & der Soziologe & der Löwe & der Rabe & der Schimpanse \\
        \hline
        Type 2 & der Elef\underline{ant} & der Emigr\underline{ant} & der Präsid\underline{ent} & der Kommun\underline{ist} \\
        & der Kapital\underline{ist} & der Astron\underline{aut} & der Diplom\underline{at} & der Kandid\underline{at} \\
        \hline
        Type 3 & der Bauer & der Bär & der Held & der Mensch \\
        & der Nachbar & der Pilot & der Idiot & der Architekt
\end{tblr}
\end{table}

\subsubsection{名词构词法}

\paragraph{使用-er描述职业}

描述从事某一类职业的人的名词通常是阳性的。

大多数情况下是通过在动词或名词词干后添加「-er」来表示,同时这些名词复数没有词尾变化 (die Musik => der Musiker => die Musiker)

但也有一些名词不是通过「-er」构成,同时它们的复数形式没有规律 (der Arzt => die Ärzte)

如果要表示从事这些直接的女性,通常在词干的单数形式后再添加「-in」,复数形式后再添加「-innen」 (die Musikin, die Musikinnen; die Ärztin, die Ärztinnen)

但是为了更中性的描述从事某一职业的人,除了可以用 Musiker/in 这样的方式来表示,也可以用这一动词的过去分词表示「做...的人」。但是德语缺少描述具有多种性别人员的群体的相关的名词,只能用「Liebe Genossen und Genossinnen(亲爱的男同志和女同志们)」类似的方法描述。

\paragraph{使用-ung名词化动词}

使用「-ung」得到的名词通常是阴性的,复数时再添加「-en」后缀。需要注意的是这与动词的动名词不同。

\paragraph{使用-h/keit名词化形容词}

使用「-heit」或「-keit」名词化形容词与英文中使用「-ness」名词化形容词的行为非常相像,这样得到的名词通常也是阴性的,复数也是再添加「-en」后缀。

\paragraph{名词的指小词}

指小词是描述更小/年轻/可爱的某一名词,通常带有非正式、喜爱的感觉,如汉语的「花=>花儿」「狗=>小狗=>狗狗」,但同汉语一样,使用一些词的指小词会显得做作或傻气。

德语中不同方言使用的指小词略有不同,一般情况下使用「-chen」或「-lein」。名词的指小词是中性名词,且复数没有变化,并且与原名词相比通常带有元音音变。

有些名词既可以使用「-chen」也可以用「-lein」,相比下可能某一种更正式,但这没有统一的规则,仅作了解即可。

\begin{table}[H]
    \caption{构成名词的方法}
    \label{tab:form-nouns}
    \centering
\begin{tblr}{
    width=\textwidth,
    colspec={cllll},
    columns={colsep=5.5pt},
    vline{2}={},
}
    构成方法& 原始形态  & 单数    & 复数    & 含义 \\
    \hline
    \SetCell[r=4]{m,c}{描述职业\\(规则)} & \SetCell[r=2]{m,l}{die Musik} & der Musik\uline{er} & dieMusik\uline{er} & \SetCell[r=2]{m,l}{音乐->音乐家} \\
    &       & die Musiker\uline{in} & die Musiker\uline{innen} &  \\
    \cline{2-5}
    & \SetCell[r=2]{m,l}{lehren} & der Lehr\uline{er} & die Lehr\uline{er} & \SetCell[r=2]{m,l}{教学->教师} \\
    &       & die Lehrer\uline{in} & die Lehrer\uline{innen} &  \\
    \hline
    \SetCell[r=4]{m,c}{描述职业\\(不规则)} & \SetCell[r=2]{m,c}{-}     & der Arzt & die Ärzte & \SetCell[r=2]{m,l}{医生} \\
    &       & die Ärzt\uline{in} & die Ärzt\uline{innen} &  \\
    \cline{2-5}
    & \SetCell[r=2]{m,c}{-}     & der Matrose & die Matrosen & \SetCell[r=2]{m,l}{水手} \\
    &       & die Matros\uline{in} & die Matros\uline{innen} &  \\
    \hline
    \SetCell[r=3]{m,c}{动词名词化} & wohnen & die Wohn\uline{ung} & die Wohn\uline{ungen} & 居住->住所 \\
    & zahlen & die Zahl\uline{ung} & die Zahl\uline{ungen} & 支付->支付行为 \\
    & regieren & die Regie\uline{ung} & die Regier\uline{ungen} & 统治->政府 \\
    \hline
    \SetCell[r=3]{m,c}{形容词名词化} & krank & die Krank\uline{heit} & die Krank\uline{heiten} & 生病的->疾病 \\
    & möglich & die Möglich\uline{keit} & die Möglich\uline{keiten} & 可能的->可能性 \\
    & schwierig & die Schwierig\uline{keit} & die Schwierig\uline{keiten} & 困难的->困难 \\
    \hline
    \SetCell[r=3]{m,c}{名词指小词} & der Tisch & das Tisch\uline{lein} & die Tisch\uline{lein} & 桌子->小桌子 \\
    & die Maus & das Mäus\uline{chen} & die Mäus\uline{chen} & 老鼠->小老鼠 \\
    & das Brot & das Bröt\uline{chen} & die Bröt\uline{chen} & 面包->面包卷 \\
\end{tblr}
\end{table}


\subsection{冠词}

德语的冠词有两个,即定冠词(der)与不定冠词(ein),虽然存在认为 kein 之类不定代词的也是冠词的观点,但根据 Duden 词典,它们并不是,德语中的冠词只有两个。冠词需要根据修饰名词的性、数、格变化。

定冠词遵循的变格规律是变格表 var1 (ein-),不定冠词遵循的变格规律基本是变格表 base (除了中性 N、A 格的是 das 而不是 des)。

简单的举一个例子:

Der Schüler(N) gab dem Lehrer(D) seinen Bericht(A).

The student(N) gave the teacher(D) his report(A).

\subsection{数词}

\subsubsection{不定数词}

\begin{itemize}
    \item viele, wenige, einige, manche, mehrere 按照变格表base变格,通常作定语,修饰复数名词

    \example{In \uline{viele} große Städte ist er in den letzten Jahren gereist.}{最近几年,他到了很多大城市去旅游。}

    \example{Mit Hilfe \uline{weniger} Bekannter hat er das Kind gefunden.}{靠少数几个熟人的帮助,他找到了那个孩子。}

    \item viel 和 wenig 可修饰零冠词的单数名词

    \example{Er hat \uline{viel} Zeit darauf verwendet.}{他把很多时间花在了这件事情上。}

    \example{Er hat uns mit \uline{wenigem} Erfreulichen überrascht.}{他难得有令人高兴的事情让我们吃惊。}

    \item alle 修饰复数名词,表示全部,但也可以修饰单数名词。如果修饰有冠词或代词的名词,无变格

    \example{Er hat \uline{alle} Leute im Büro ins Kino eingeladen.}{他请办公室所有的人去看电影。}

    \example{\uline{Aller} Anfang ist schwer.}{万事开头难。}

    \example{Ich mache das mit all meiner Kraft.}{我尽自己的一切力量做这件事。}

    \item ein bisschen: 一点,ein paar: 几个,ein wenig: 少许,etwas: 一些,genug: 足够,mehr: 较多,nichts: 没什么。其中,除了 ein paar 修饰复数名词,mehr 不限,其他需要修饰单数名词,不变格

    \example{Ich möchte noch etwas Zucker.}{我还想要一些糖。}

    \example{Er hat mit \uline{ein paar} Bekannten geredet.}{他跟几个熟人谈过。}

    \example{Er hat \uline{mehr} Freizeit als ich.}{他的空闲时间比我多。}

\end{itemize}

\subsubsection{基数词}

\begin{itemize}
    \item \SIrange{0}{12}{}:基本数词
    \item \SIrange{13}{19}{}:基本数词+ zehn
    \item \SIrange{20}{99}{}:$10$的倍数为「基本数词+ zig」,其他为「个位数+ und +十位数」
    \item \SIrange{100}{999}{}:$100$的倍数为「基本数词+ hundert」;口语中可省略 ein(如果百位为$1$)
    \item \SIrange{1000}{999999}{}:$1000$的倍数为「基本数词+ tausend」;口语中可省略 ein(如果最高位为$1$);口语中可省略连接的 und
    \item \num{1d6},\num{1d9},\num{1d12},\num{1d15}:用 Million, Milliarde, Billion, Billiarde 表示,均为阴性,超过一个单位使用复数,如 zwei Millionen
\end{itemize}
\begin{table}[htbp]
    \centering
\begin{tblr}{
    width=\textwidth,
    colspec={rlrlrlrl},
    columns={colsep=2pt},
    vline{3,5,7},
}
0 & null  & 10 & zehn     & 20 & \uleach{zwan,zig}        & 100    & \uleach{(ein),hundert}                     \\
1 & eins  & 11 & elf      & 21 & \uleach{ein,und,zwanzig}  & 102    & \uleach{(ein),hundert,und,zwei}                     \\
2 & zwei  & 12 & zwölf    & 22 & \uleach{zwei,und,zwanzig} & 110    & \uleach{(ein),hundert,zehn}                 \\
3 & drei  & 13 & \uleach{drei,zehn} & 30 & \uleach{dreiß,ig} & 122    & \uleach{(ein),hundert,zwei,und,zwanzig}       \\
4 & vier  & 14 & \uleach{vier,zehn} & 40 & \uleach{vier,zig} & 500    & \uleach{fünf,hundert}        \\
5 & fünf  & 15 & \uleach{fünf,zehn} & 50 & \uleach{fünf,zig} & 1000   & \uleach{(ein),tausend}       \\
6 & sechs & 16 & \uleach{sech,zehn} & 60 & \uleach{sech,zig} & 1012   & \uleach{(ein),tausend,(und),zwölf}    \\
7 & siben & 17 & \uleach{sieb,zehn} & 70 & \uleach{sieb,zig} & 1100   & \uleach{(ein),tausend,hundert}       \\
8 & acht  & 18 & \uleach{acht,zehn} & 80 & \uleach{acht,zig} & 8000   & \uleach{ach,tausend}         \\
9 & neun  & 19 & \uleach{neun,zehn} & 90 & \uleach{neun,zig} & 121000 & \uleach{(ein),hundert,ein,und,zwanzig,tausend} \\
\end{tblr}
\end{table}

\subsubsection*{基本数词的变格}

\begin{itemize}
    \item eins作数词或不定代词:作数词时位于名词前,作代词单独使用,变格都同不定冠词变格一致,口语中需重读。作不定代词时,中性 N、A格可为ein-s或ein-es

    \example{Ich habe gerade einen Brief aus Deutschland erhalten.}{我刚收到从德国来的一封信。}

    \example{Die Messe findet in einer dieser Großstädte statt.}{博览会将在这些大城市中的一个举行。}

    \item eins 位于定冠词/代词后:按照变格表var1变格

    \example{Die eine Wohnung ist heute verkauft.}{这一套房子今天卖出去了。}

    \item ein bis/oder (zwei),表示时间,计算或单独读数时:不发生变格

    \example{Das Päckchen hat ein Gewicht von ein bis zwei Kilogramm.}{这个包裹一到两公斤重。}

    \item ein 位于复合数词结尾:需要变格,如果位于复数名词前,eins变为ein

    \example{Er hat ein Gewicht von hundert und einem Kilogramm.\\Er hat ein Gewicht von hundertundeinem Kilogramm.}{他重一百零一公斤。}

    \example{Er hat schon hundert und \uline{eine} \uline{Seite} gelesen.\\Er hat schon hundertund\uline{ein} \uline{Seiten} gelesen.}{他已经读了一百零一页。}

    \item zwei 和 drei 在无冠词名词前使用 G 格词尾添加 -er

    \example{die Beziehung zweier Länder.\\$\hookrightarrow$die Beziehung von zwei Ländern.}{两个国家的关系。}
    
    \item beide(-s) 代替提到的两者,按照形容词变格
    
    \example{Es freut uns beide, dass wir in diesem Jahr nach Deutschland fahren kömmen.}{我们两个很高兴今天能到德国。}

    \example{Dieses beides haben wir erledigt.}{这两件事我们都解决了。}

    \example{Ihr sollt \uline{beiden} heute Nachmittag zu mir kommen.\\
    $\hookrightarrow$Ihr \uline{beiden} sollt heute Nachmittag zu mir kommen.}{你们两个今天下午到我这里来。}
\end{itemize}

\subsubsection*{基数词的用法}

\begin{itemize}

    \item 表达年份:使用 im Jahr(e) + 年份数字,通常情况用德语书写使用 hundert 构成,但如果百位数为零,使用 tausend 构成

    im Jahre 1976 / \uleach{neunzehn,hundert,sechs,und,sibzig}

    im Jahre 2007 / \uleach{zwei,tausend,sieben}

    \item 公元前后

    v. Chr. G. (vor Christi Geburt) / v. Chr. (vor Christus) / v. u. Z (vor unserer Zeitrechnung)

    n. Chr. G. (nach Christi Geburt) / n. Chr. (nach Christus) / u. Z (unserer Zeitrechnung)

    im Jahre 75 v. Chr. G. / von 206 v. u. Z. bis 220 u. Z.

    \item 数学表达式

    \begin{tblr}{
        colspec={rl},
        rows={m,rowsep=0.5pt},}
        $6+7=13$ & {Sech und/plus sieben ist/gleich dreizehn.} \\
        $9-5=4$ & {Neun weniger/minus fünf ist/gleich vier.} \\
        $3\times 6=18$ & {Drei [mal/multipliziert mit] sechs ist achtzehn.} \\
        $24\div 3=8$ & {Vierundzwanzig [durch / geteilt durch / dividiert durch] drei ist acht.} \\
        $9^2=81$ & {Neun hoch zwei ist einundachtzig.} \\
        $5^3=125$ & {Fünf hoch drei ist hundertfünfundzwanzig.} \\
        $\sqrt{64}=8$ & {Quadratwurzel aus vierundsechzig ist acht.} \\
        $\sqrt[3]{64}=4$ & {Kubikwurzel aus vierundsechzig ist vier.} \\
        $y=f(x)$ & {Ypsilon ist Funktion $f$ von $x$.} \\
    \end{tblr}

    \item 表达价格:价格如果高于一欧元,可以省略欧分单位,低于一欧元,欧分单位不可省略

    \begin{tblr}{
        colspec={rl},
        rows={m,rowsep=0.5pt},}
        $25,16$ \texteuro & fünfundzwanzig Euro sechzehn \\
        $199,98$ \texteuro & hundertneunundneunzig Euro achtundneuzig \\
        $0,98$ \texteuro & achundneunzig Cent \\
    \end{tblr}

    \item 表达时间:分为官方和日常用法,回答 wie spät ist es? 或 wie viel Uhr ist es?。官方用法使用24小时计时,先报小时,再报分钟;非官方用法通常使用12小时,基于整点或半点,搭配 vor/nach 表达。口语中还可以使用副词说明上下午,也可以省去 Uhr

    \begin{tblr}{
        colspec={rll},
        % rows={m},
        rows={m,rowsep=2pt},
    }
    \textbf{时间} & \textbf{官方用法} & \textbf{日常用法} \\
    \hline
    08:00 & acht Uhr & acht (Uhr) (morgen) \\
    08:09 & acht Uhr neun & neun (Minuten) nach acht (morgen) \\
    08:15 & acht Uhr fünfzehn & {Viertel nach acht (morgen)\\fünfzehn (Minuten) nach acht (morgen)} \\
    08:27 & acht Uhr sibenundzwanzig & {drei (Minuten) vor halb neun (morgen)\\siebenundzwanzig (Minuten) nach acht (morgen)} \\
    08:30 & acht Uhr dreißig & halb neun (morgen) \\
    08:37 & acht Uhr siebenunddreißig & {sieben (Minuten) nach halb neun (morgen)\\dreiundzwanzig (Minuten) vor neu (morgen)n} \\
    08:45 & acht Uhr fünfundvierzig & {Viertel vor neun (morgen)\\fünfzehn (Minuten) vor neun (morgen)} \\
    08:53 & acht Uhr dreiundfünfzig & sieben (Minuten) vor neun (morgen)\\
    12:00 & zwölf Uhr & zwölf Uhr (Mittag) \\
    20:00 & acht Uhr & acht (Uhr) (abends) \\
    20:09 & acht Uhr neun & neun (Minuten) nach acht (abends) \\
    20:15 & acht Uhr fünfzehn & {Viertel nach acht\\fünfzehn (Minuten) nach acht (abends)} \\
    20:27 & acht Uhr sibenundzwanzig & {drei (Minuten) vor halb neun (abends)\\siebenundzwanzig (Minuten) nach acht (abends)} \\
    20:30 & acht Uhr dreißig & halb neun (abends) \\
    20:37 & acht Uhr siebenunddreißig & {sieben (Minuten) nach halb neun (abends)\\dreiundzwanzig (Minuten) vor neun (abends)} \\
    20:45 & acht Uhr fünfundvierzig & {Viertel vor neun (abends)\\fünfzehn (Minuten) vor neun (abends)} \\
    20:53 & acht Uhr dreiundfünfzig & sieben (Minuten) vor neun (abends)\\
    {00:00\\24:00} & vierundzwanzig Uhr & {zwölf Uhr (Abends)\\Mitternacht} \\
    \end{tblr}
\clearpage
\end{itemize}

\subsubsection{序数词}

序数词有基数词 + 词尾「-t / -st」构成,在单数/复数名词前有词尾变化,用来回答 der wievielte 的问题。书面表达使用阿拉伯数字加点,或字母表达(25. / der zweiundfünfzigste)。
\begin{table}[H]
    \centering
    \begin{tblr}{
        colspec={rlrlrl},
        rows={m,rowsep=0.5pt},}
        1.  & erste   & 11. & elfte      & 21.      & enundzwanzigste   \\
        2.  & zweite  & 12. & zwölfte    & 22.      & zweiundzwanzigste \\
        3.  & dritte  & 13. & dreizehnte & 23.      & dreiundzwanzigste \\
        4.  & vierte  & 14. & vierzehnte & 30.      & dreißigste        \\
        5.  & fünfte  & 15. & fünfzehnte & 40.      & vierzigste        \\
        6.  & sechste & 16. & sechzehnte & 50.      & fünfzigste        \\
        7.  & siebst  & 17. & siebzehnte & 100.     & hundertste        \\
        8.  & achte   & 18. & achtzehnte & 101.     & hunderterste      \\
        9.  & neunte  & 19  & neunzehnte & 10000.   & zehntausendste    \\
        10. & zehnte  & 20. & zwanzigste & 1000000. & millionste        \\
    \end{tblr}
\end{table}

序数词用法如下:
\begin{itemize}
    \item 序数词前加定冠词,按形容词变格

    \example{Ich bin schon zum zweiten Mal dorthin gegangen.}{我已经第二次去那里了。}

    \example{In dem vierten Haus wohnt ein Alter.}{在第四栋房子里住着一位老人。}

    \item 表示日期

    \example{-Der vievielte ist heute?\\-Heute ist der [08.06 / achte sechste / achte Juni].}{-今天几号?\\-今天6月8日。}

    \example{Vom 5. Juni bis zum 1. September haben die Studenten Ferien.}{学生们的假期从7月5日到9月1日。}

    \item 与 zu 连用表示方式,词尾不变化

    \example{Wir machen diese Übungen zu zweit.}{我们两个人来做这些练习。}

    \example{Die Schüler marchierten zu viert in einer Reihe vom Schulhof ab.}{学生们四人一排列队从校园出发。}

    \item 作名词

    \example{In dem Schloss wohnte König Ludwig der Zweite.}{Ludwig 二世国王曾经住在这座王宫里。}

    \example{Er war beim Wettkampf Dritter.}{竞赛中他得了第三名。}

    \item 与不定代词 jed- 连用,表示「每\ldots 」

    \example{jede dritte Woch / alle drei Wochen}{每三周}

    \example{Jeder fünfte Deutsche gehört zu einem Verein.}{每五个德国人中就有一人是协会成员。}

\end{itemize}

\subsubsection{分数词}
分数词(die Bruchzahlen)表示整体的一部分,分子为基数词,分母为序数词 +「-el」
\begin{table}[H]
    \centering
    \begin{tblr}{
        colspec={rlrl},
        rows={m,rowsep=0.5pt},}
        0,5   & null Komma fünf   & $1\frac12$ & eineinhalb / anderthalb     \\
        3,125 & drei Komma eins zwei fünf  & $2\frac12$ & zwei(und)einhalb            \\
        $\frac12$ & ein halb  & $3\frac14$ & drei(und)einviertel         \\
        $\frac13$ & ein Drittel  & $4\frac34$ & vier(und)dreiviertel        \\
        $\frac14$ & ein Viertel  & $5\frac{3}{10}$ & fünf(und)dreizehntel        \\
        $\frac23$ & zwei Drittel & $6\frac{7}{100}$ & sechs(und)siebenhundertstel \\
        $\frac{1}{20}$ & ein Zwanzigstel  & $7\frac{9}{100}$ & sieben(und)neuntausendstel  \\
    \end{tblr}
\end{table}

分数词的用法如下:
\begin{itemize}
    \item 作定语,不变格,大小写均可。如果 halb- 作形容词定语,需要变格,名词形式为 die Hälfte。在表示计量单位的名词后用单数,其他单位前,如果分子为1,该名词使用单数,反之使用复数
    
    \example{Gestern hat er drei \uline{viertel} Liter Bier getrunken.}{昨天他喝了四分之三升的啤酒。}

    \example{Ich bin in Deutschland für \uline{ein halbes} Jahr gewessen.}{我在德国待了半年。}

    \example{-Wie lange sin Sie hier in Mainz?\\-Ich bin hier \uline{anderthalb} Jahre.}{-您在 Mainz 待了多长时间了?\\-我在这里一年半了。}

    \example{ein drittel Jahr Arbeit\\dreieinhalb Jahre Aufenthalt}{四个月的工作\\三年半的居留}

    \item 单独作中性名词使用,也可以和名词一起构成复合名词
    
    \example{Ich gebe \uline{ein Drittel} meines Lohnes für die Wohnungsmiete aus.}{我收入的三分之一用来支付房租。}

    \example{Er will ein \uline{Viertel} Liter Rotwein.}{他要四分之一升红酒。}

    \example{Dieser Prozess dauert nur eine \uline{Zehntelsekunde.}}{这个过程只持续十分之一秒。}

\end{itemize}

\subsubsection{其他数词}

\begin{itemize}

    \item 分配数词(die Einleitungszahlen)表示叙述的顺序,由序数词 + 词尾「-stens」构成,无词尾变化,作副词,主要用于列举项目
    
    \example{Ich nehme Taxi, denn \uline{erstens} regnet es und \uline{zweitens} habe ich keinen Schirm.}{我们坐出租车,因为第一在下雨,第二我没有带伞。}

    \example{Das Gerät ist \uline{1.} presgünstig, \uline{2.} nicht sehr groß und \uline{3.} einfach zu bedienen.}{这个机器第一便宜,第二个头不大,第三操作简单。}

    \item 重复数词(die Wiederholungszahlen) 说明重复的次数,回答 wie oft? 或 wie vielmal?,由基数词或不定数词 + 词尾「-mal(s)」构成。重复数词词尾加「-ig」,在名词前作定语或单独使用
    
    \example{\uline{Einmal} im Jahre besuchen wir die Großeltern in der Heimat.}{每年我们回家乡看望祖父母一次。}

    \example{\uline{Einmal} ist keinmal.}{一次不算数。}

    \example{\uline{Tausendmal} Entschuldigung, dass ich zu spät komme.}{我因为迟到,上千次地表示道歉。}

    \example{-Wie \uline{vielmal} / oft sind Sie in Deutschland gewesen?\\-Ich bin schon \uline{dreimal} dort gewesen.}{-您到过德国几次?\\-那里我已经去过三次了。}

    \example{Nach \uline{mehrmaligen} Diskussionen ist es endlich gelungen, eine gemeinsame Entscheidung zu treffen.}{经过多次讨论终于成功地达成了一致的决定。}

    \example{Wir haben das \uline{niemals} gesehen.}{我们从来没见过这样的东西。}

    \item 倍数词(die Vervielfältigungszahlen)用来表示相同事物或动作的倍数,由基数词/不定数词 + 词尾「-fältig」构成。位于名词前作形容词定语需要变格;作副词则无变格。同时,重复数词和倍数词的区别主要在于重复数词说明时间上的连续重复,倍数词说明同时刻的共存。特别地,zweifach 可以用 doppelt 表示,einfach 也有不同的含义
    
    \example{Packen Sie bitte eine \uline{dreifache} Menge davon.}{请您把这个按三倍的量包装起来。}

    \example{-Wie oft muss ich das Formular ausfüllen?\\-Füllen Sie bitte es \uline{dreifach} aus!}{-这张表格我要填几份?\\-请您填写三份。}

    \example{der \uline{dreimalige} Sieg des Sportlers\\der \uline{dreifache} Sieg des Sportlers}{这个运动员先后三次的胜利\\这个运动员的三项胜利}

    \example{Der Koffer hat einen \uline{doppelten/zweifachen} Boden.}{这个箱子有一个双层底。}

    \example{eine \uline{einfache} Kleidung\\eine \uline{einfache} Aufgabe}{朴素的衣服\\简单的任务}

    \item 种类数词(die Gattungszahlen) 说明不同的种类数、数目和形式,由基数词或不定数词 + 词尾「-erlei」构成。可以作形容词,或同系动词连用
    
    \example{Er hat \uline{achterlei} zu tun.}{他有八件事情要做。}

    \example{Sie können Ihren Job auf \uline{vielerlei} Arten erledigen.}{您可以通过各种方式解决您的工作问题。}

    \example{Der Regenbogen leuchtet in \uline{siebenerlei} Farben.}{彩虹发出七彩光芒。}

    \example{Verhalten und Halten ist \uline{zweierlei}.}{许下诺言和遵守诺言是两回事。}

    \item 划分数词(die Verteilungszahlen) 说明事物的平均划分,由基数词 + 副词 je 构成
    
    \example{Es wir hier in drei Schichten zu \uline{je acht} Stunden gearbeitet.}{这里分三班工作,每班八小时。}

    \example{In dieser Stadt konnten sie als Arbeitslose vom Arbeitsamt \uline{je 680} Euro erhalten.}{在这个城市里,作为失业者的他们当时可以从劳动局那里每人得到 680 欧元。}

\end{itemize}

\subsection{代词}

德语中的代词主要有人称代词及其衍生出的反身代词与物主代词、不定代词、指示代词、疑问代词、关系代词。有一些词有多重身份,因此使用时要注意。

\subsubsection{人称代词}

德语中的人称代词与英语的不同点主要如下:
\begin{enumerate}[leftmargin=3.5em, topsep=0pt, itemsep=0pt, parsep=0pt]
    \item 德语第二人称的单复数形式不一样
    \item 德语第二人称增加了礼貌用法 Sie,单复数同型,且与第三人称复数具有一样的变格规律,只是首字母大小写不同
\end{enumerate}

如\cref{tab:pronoun-declension},人称代词的属格形式很少使用,它们由物主代词的词根加上 er 可以得到,再根据代指内容的性,可以按照变格表var1的形式进行变格。

反身代词只有宾格和与格形式,除了第三人称的宾/与格都使用 sich 之外,其他人称的宾/与格形式与人称代词的宾/与格形式相同。

\begin{table}[htbp]
    \caption{人称代词的变格}
    \label{tab:pronoun-declension}
    \centering
\begin{threeparttable}
\begin{tblr}{
    width=\textwidth,
    rows={m},
    column{1}={wd=1cm,c},
    cell{2}{1}={r=4}{c,m},
    vline{3}={},
}
    & & I     & you   & he    & it    & she   & we    & you   & {they/\\you(polite)} \\
    \hline
    人称代词 & N     & ich   & du    & er    & es    & sie   & wir   & ihr   & Sie \\
    & A     & mich  & dich  & ihn   & es    & sie   & uns   & euch  & Sie \\
    & D     & mir   & dir   & ihm   & ihm   & ihr   & uns   & euch  & Ihnen \\
    & G\tnote{1} & mein & dein & sein & sein & ihr & uns & euer  & Ihr \\
    \hline
    {反身\\代词} & A/D   & …     & …     & sich & sich & sich & …     & …     & Sich \\
\end{tblr}
\begin{tablenotes}
    \item[1] 这里实际上是物主代词词根的形式,真正人称代词的属格形式(G)需要在词尾加上「-er」(除了 euer 不需要)
\end{tablenotes}
\end{threeparttable}
\end{table}

\subsubsection{不定代词}

德语中的不定代词(Indefinitpronomen)用于代指身份/数量不明的人/物。

etwas nichts alles was([irgend]etwas),只有单数形式,不发生变格。

all-, jed-, beid-viel-, manch-, wer([irgend]jemand),按照变格表base变格。

kein, man, jemand, niemand, einige-, wenige-,按照变格表var1变格。

irgend- 修饰的按照被修饰词变格。

man 除主格外的变格与 ein 相同。

jed- 的复数变格与 all- 相同。

\subsubsection{指示代词}

德语中的指示代词(Demonstrativpronomen)包括 der,dieser,jener,derjenige,derselbe,solcher,selber,可以按变格分成三类

der 作为指示代词,变格形式与作定冠词的略有不同,且没有准确的对应译法,应根据上下文。「TODO deren derer 区别用法」

dieser(this/these),jener(that/those),solcher(such) 按照变格表base变格,er是词尾

derjenige,derselbe分成两部分,der-部分按照变格表base变格(定冠词),-jenige和-selbe 按照变格表var2变格(弱变化形容词)

der ein 与 der anderen 也可以认为是指示代词,变格形式同 derjenige,通常搭配使用,表示「一个...另一个」(有复数形式)。

selbst 与 selber 不变格,更偏向书面语言使用,表示自身。

\begin{table}[htbp]
    \caption{指示代词的变格}
    \label{tab:demonstrativpronomen-declensionn}
    \centering
\begin{tblr}{
    width=\textwidth,
    colspec={c|llll},
    cell{5}{2-Z}={markyellow},
    cell{4}{Z}={markyellow},
}
  & m.         & n.         & f.         & pl.         \\
\hline
N & der        & das        & die        & die         \\
A & den        & das        & die        & die         \\
D & dem        & dem        & der        & denen       \\
G & dessen     & dessen     & deren      & deren/derer \\
\hline
N & dieser     & dieses     & diese      & diese       \\
A & diesen     & dieses     & diese      & diese       \\
D & diesem     & diesem     & dieser     & diesen      \\
G & dieses     & dieses     & dieser     & dieser      \\
\hline
N & derjenige  & dasjenige  & diejenige  & derjenigen  \\
A & denjenigen & dasjenige  & dierjenige & derjenigen  \\
D & demjenigen & demjenigen & derjenigen & denjenigen  \\
G & desjenigen & desjenigen & derjenigen & derjenigen 
\end{tblr}
\end{table}
\subsubsection{疑问代词}

德语中的疑问代词(Interrogativpronomen)包括 wer(who)、was(what)、welch-(which) 和 was für ein(which kind of),其中 welch 需要根据代指对象的性按照变格表base变格,wer 和 was 没有性和数的变化,was für ein 只有 ein 按照不定冠词变格。

\begin{table}[htbp]
    \caption{疑问代词的变格}
    \label{tab:interrogativpronoun-declension}
    \centering
\begin{tblr}{
    width=\textwidth,
    colspec={c|l|l|llll},
}
          & -      & -     & m.      & n.      & f.      & pl.     \\
    \hline
    N     & wer    & was   & welcher & welches & welche  & welche  \\
    A     & wen    & was   & welchen & welches & welche  & welche  \\
    D     & wem    & -     & welchem & welchem & welcher & welchen \\
    G     & wessen & -     & welches & welches & welcher & welcher \\
\end{tblr}
\end{table}

\subsubsection{关系代词}\label{sec:relativnoun}

德语中的关系代词(Relativpronomen)主要用于引导从句。

der 的变格与指示代词 der 一样。其引导的从句只能用做名词/代词的定语

wer 标识泛指的人,只能与 der 连用。

was 标识泛指的物,常与 das 搭配,可以用作中性不定代词或整句的关系代词,相关词可以使 das/etwas/alles/nichts/manches/vieles/weniges,也可与中性的名词化形容词最高级使用。

welch- 用于避免重复使用 der-,起到替代的作用,变格与疑问代词 welch- 一样使用变格表base,但没有 G 格变化。

\subsection{形容词}

\subsubsection{形容词的变格}

形容词的变格可以分成三类,强变化、弱变化、混合变化。

注意,形容词作表语时不变格,只有直接修饰名词时才发生变格。形容词采取何种变化方式(强/弱/混合)不是形容词的性质,而是由形容词修饰的名词及其冠词决定的。

总的来说,形容词的变格非常的简单,概括下来就是{\bf 当冠词或名词发生了强变化,形容词弱变化,其他时候形容词强变化}:

\begin{enumerate}[leftmargin=3.5em, topsep=0pt, itemsep=0pt, parsep=0pt]
    \item 形容词如果要变格,要么承担强变化词尾,要么承担弱变化词尾。强变化词尾来自变格表base,弱变化词尾来自变格表var2
    \item 如果形容词修饰的名词的冠词强变化,那么形容词弱变化。这有两重含义
    \begin{enumerate}
        \item 当冠词是定冠词或{\bf 使用变格表base变化的}(见代词一节),那么形容词全部弱变化(这也称为「形容词的弱变化」)
        \item 当冠词是不定冠词或物主代词或 {\bf 使用变格表var1变化的}(见代词一节),那么除了单数主格阳性、单数主格宾格中性时,其他位置上发生弱变化(这也称为「混合变化」)
    \end{enumerate}
    \item 如果形容词修饰的名词无冠词,那么应有形容词承担所有强变化词尾,也就是按照变格表base变化。但是在单数属格的阳性、中性、复数与格这三个位置上,名词发生了强变化,因此这三个位置上形容词弱变化(这也称为「强变化」)
\end{enumerate}

实际上形容词的变格就像过筛子,只要冠词或者名词用了表base(强变化),那么形容词弱变化,只不过一部分冠词或者代词在黄色位置没有变化,而名词在绿色位置变化了,因此产生了看似复杂的变化。{\bf 只要记住了表base和表var2,并且记住了与不定冠词变格形式相同的代词有哪些,就没什么难的}。

\subsubsection{形容词的变级}

德语形容词的变级与英语比较相似。当形容词不变格时,比较级添加「-er」,最高级形容词前添加「am」词尾添加「-sten」。

Mein Auto ist klein.

Dein Auto ist klein\e{er}.

Sein Auto ist \e{am} klein\e{sten}.

当形容词既要变级也要变格时,以变级后的词去掉词尾「-r」或「-n」作为词干,按照变格规则变格。

Dein groß\uline{e} Apfel.

Nimm das größ\e{er}\uline{e} Stück.

Mein größ\e{t}\uline{\e{e}r} Sieg was...

以「-el」或「-er」结尾的形容词变级时词干的尾部去掉「-e-」但保留辅音。以「-t / d / s / z / ß / sch / x」结尾的词在最高级时添加「-e-」以改善拼读。形容词在变级时可能发生元音音变,这些音变是不规则的,但是越短 (单音节) 越常用的形容词更容易发生音变。

德语也有一些形容词的变级是完全不规则的,但并不多,需要单独记忆。

德语的形容词总可以应该按照上述规则变级,而不像英语中的多音节词使用 more/most 变级。

\subsubsection{形容词的使用}

so... wie... 与... 一样

genauso(ebenso)... wie... 与... 完全一样

Er ist so alt wie du.

---

als = than

Meine Wohnung ist größer als deine Wohnung.

immer = more and more,noch = more,viel = much more,eher = more


