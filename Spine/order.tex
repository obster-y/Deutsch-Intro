\section{造句}


\subsection{分区结构和句框}
\begin{figure}[H]
    \centering
    
\tikzset{
    ncbar angle/.initial=90,
    ncbar/.style={
        to path=(\tikztostart)
        -- ($(\tikztostart)!#1!\pgfkeysvalueof{/tikz/ncbar angle}:(\tikztotarget)$)
        -- ($(\tikztotarget)!($(\tikztostart)!#1!\pgfkeysvalueof{/tikz/ncbar angle}:(\tikztotarget)$)!\pgfkeysvalueof{/tikz/ncbar angle}:(\tikztostart)$)
        -- (\tikztotarget)
    },
    ncbar/.default=0.5cm,
}

\tikzset{square left brace/.style={ncbar=0.12cm}}
\tikzset{square right brace/.style={ncbar=-0.12cm}}

\begin{tikzpicture}[
    every node/.style={align=center, inner sep=1mm},
    node distance=0.25cm,
    align = center,
    ]

    \begin{scope}[xshift=0cm, yshift=0cm]
        \node (VF) [inner sep=0mm] {前区\\(Vorfeld)};
        \node (LSK) [right=of VF,draw] {左框\\(linke Satzklammer)};
        \node (MF) [right=of LSK,inner sep=0mm] {中区\\(Mittelfeld)};
        \node (RSK) [right=of MF,draw] {右框\\(rechte Satzklammer)};
        \node (NF) [right=of RSK,inner sep=0mm] {后区\\(Nachfeld)};
        \draw [black, thick] (VF.south west) to [square left brace] (VF.north west);
        \draw [black, thick] (VF.south east) to [square right brace] (VF.north east);
        \draw [black, thick] (MF.south west) to [square left brace] (MF.north west);
        \draw [black, thick] (MF.south east) to [square right brace] (MF.north east);
        \draw [black, thick] (NF.south west) to [square left brace] (NF.north west);
        \draw [black, thick] (NF.south east) to [square right brace] (NF.north east);
        \draw [decorate, decoration={brace, raise=0.2cm, amplitude = 0.5cm}] (VF.north) to node[above=0.7cm] {分区结构(Feldermodell)} (NF.north);
        \draw [decorate, decoration={brace, raise=0.1cm, amplitude = 0.3cm, mirror}] (LSK.south) to node[below=0.4cm] {句框(Satzklammer)} (RSK.south);
    \end{scope}
\end{tikzpicture}
    % \caption{动词的分类}
    % \label{fig:verb-categories}
\end{figure}

大部分语言造句时的核心在于谓语,德语的谓语具有「分区结构」(Feldermodell),分区结构也和「句框」(Satzklammer)有紧密联系。

句框的核心是将谓语分成「变位部分」与「非变位部分」,将变位部分放在句框首部(左框/LSK),非变位部分放在尾部(右框/RSK)。

需要按照句框构造的谓语可以有以下这些情况,加号左边为发生变位的部分(LSK),右边为不变位的部分(RSK):
\begin{itemize}
    \item「werden/haben/sein + 实义动词」
    \item 「情态动词/其他动词 + 不定式」
    \item 「词干 + 可分前缀」
    \item 「系动词 + 表语」
    \item 「功能动词 + 名词性部分」
    \item 「动词 + nicht」
\end{itemize}

\subsection{前区成分}
VF 通常包含主语,或表示时间/元音/状态/地点的说明语。如果句子的其他成分位于前场,则要么其是已知信息,要么为被强调的成分。

\subsubsection{通常的前区成分}

\begin{itemize}
    \item (形式)主语或说明语

    \example{\uline{Er} hat seinem Enkel das Buch am Freitag geschenkt.}{他周五送给孙子这本书。}
    \example{\uline{Mit diesen dreckigen Händen} kommst du nicht an den Tisch.}{你的手这么脏,不能上桌。}
    \example{\uline{Am Freitag} hat er seinem Enkel das Buch geschenkt.}{周五他把书送给你孙子。}

    \item 特殊疑问词
    \item zu + 不定式结构
    \item 其他结构
    \item 某些可分前缀
    \item 从句
    \item 前区为空
\end{itemize}

\subsubsection{不可放置于前区的成分}
\begin{itemize}
    \item 反身代词
    \item 无独立意义的可分前缀
    \item 不定冠词的 A、D 格
    \item 具有连词功能的副词
    \item 单独的否定词 nicht,也就是说 nicht + 被修饰成分可以
\end{itemize}

\subsubsection{前区之前}
而还有一些词称为零占位词,放于前区之前

\subsection{中区成分与语序}

\subsubsection{中区语序}
\subsubsection{主语}
\subsubsection{宾语}
\subsubsection{其他补足语}
\subsubsection{小品词}
\subsubsection{中区语序的几个原则}


\subsection{破框和后区成分}


句框结构这一规则导致的后果就是,在有很长的从句或很多限定词时导致MF太长,需要等到MF结束后才能知道RSK究竟是什么,因此在一些情况下,MF中的一些成分可以提取出来,放到NF。这种将成分从MF移到NF的情况称为「破框」(Ausklammerung),取「打破句框」之意,实则是将一些成分放至NF。这并不是一种「固定搭配」「经验规则」,而是合规的语法变化。这种行为常见于口语,目的使对话更易理解。

\subsubsection{zu + 不定式结构}
\subsubsection{}
\subsection{从句}
