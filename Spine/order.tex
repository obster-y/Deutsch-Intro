\section{句子结构}

\subsection{句式}

句式(der Satzbauplan)是指各句子成分(主谓宾定状补)在句子的组织方式。德语在这里主要关注配价理论(die Valenztheorie)。

句子的核心成分是谓语,谓语至少由一个限定动词构成。句子的其他成分可以分为补足语(Ergänzungen)和说明语(Angaben),其中补足语分为受动词支配的一级补足语(Ergänzungen ersten Grades),受形容词支配的二级补足语(Ergänzungen zweiten Grades)。而补足语也可以分为必要性补足语(freie Ergänzung)和选择性补足语(fakultative Ergänzung),其中必要性补足语是句子结构和句义完整性必需的,选择性补足语受支配,但不是必要成分。而说明语既不受谓语支配,也不对句义或结构有影响。

\begin{table}[htbp]
    \caption{说明语和补足语}
    \label{tab:erg-ang}
    \centering
\begin{tblr}{
    width=\linewidth,
    % rows={m,abovesep=0pt,belowsep=0pt},
    colspec={X[l,0.35]X[l,0.3]X[l,0.35]},
    columns={l,colsep=2pt},
}
    \textbf{必要性补足语} & \textbf{选择性补足语} & \textbf{说明语} \\
    \hline
    {Der Vortag dauert \uline{eine halbe Stunde.}\\\sout{$\hookrightarrow$Der Vortrag dauert.}} & {Sie fahren \uline{ans Meer}.\\$\hookrightarrow$Sie fahren.} & {Er ist \uline{heute Morgen} angekommen.\\$\hookrightarrow$Er ist angekommen.}\\
    {报告持续半个小时。\\\sout{$\hookrightarrow$报告持续。}} & {他们开车去海边\\$\hookrightarrow$他们开车。} & {他今天早晨到的。\\$\hookrightarrow$他到了。}\\
    \hline
    {Wir wohnen \uline{in Berlin.}\\\sout{$\hookrightarrow$Wir wohnen.}} & {Ich esse \uline{eine Apfelsine.}\\$\hookrightarrow$Ich esse.} & {Ich esse \uline{schnell} eine Apfelsine. \\ $\hookrightarrow$ Ich esse eine Apfelsine.} \\
    {我们住在柏林。\\\sout{$\hookrightarrow$我们住。}} & {我吃橘子\\$\hookrightarrow$我吃。} & {我吃橘子很快\\$\hookrightarrow$我吃橘子。} \\
    \hline
    {Er schlug mir das Glas \uline{aus der Hand}.\\\sout{$\hookrightarrow$Er schlug mir das Glas.}} & {Wir stehen \uline{in einer Reihe}.\\$\hookrightarrow$Wir stehen.} & {Wir arbeiten \uline{schon seit langer Zeit} in Shanghai.\\$\hookrightarrow$Wir arbeiten in Shanghai.} \\
    {他从我手里打碎了玻璃杯。\\\sout{$\hookrightarrow$他打我玻璃杯。}} & {我们站成一排。\\$\hookrightarrow$我们站着。} & {我们在上海工作很久了。\\$\hookrightarrow$我们在上海工作。} \\
\end{tblr}
\end{table}

\subsection{分区结构和句框}
\begin{figure}[H]
    \centering
    
\tikzset{
    ncbar angle/.initial=90,
    ncbar/.style={
        to path=(\tikztostart)
        -- ($(\tikztostart)!#1!\pgfkeysvalueof{/tikz/ncbar angle}:(\tikztotarget)$)
        -- ($(\tikztotarget)!($(\tikztostart)!#1!\pgfkeysvalueof{/tikz/ncbar angle}:(\tikztotarget)$)!\pgfkeysvalueof{/tikz/ncbar angle}:(\tikztostart)$)
        -- (\tikztotarget)
    },
    ncbar/.default=0.5cm,
}

\tikzset{square left brace/.style={ncbar=0.12cm}}
\tikzset{square right brace/.style={ncbar=-0.12cm}}

\begin{tikzpicture}[
    every node/.style={align=center, inner sep=1mm},
    node distance=0.25cm,
    align = center,
    ]

    \begin{scope}[xshift=0cm, yshift=0cm]
        \node (VF) [inner sep=0mm] {前区\\(Vorfeld)};
        \node (LSK) [right=of VF,draw] {左框\\(linke Satzklammer)};
        \node (MF) [right=of LSK,inner sep=0mm] {中区\\(Mittelfeld)};
        \node (RSK) [right=of MF,draw] {右框\\(rechte Satzklammer)};
        \node (NF) [right=of RSK,inner sep=0mm] {后区\\(Nachfeld)};
        \draw [black, thick] (VF.south west) to [square left brace] (VF.north west);
        \draw [black, thick] (VF.south east) to [square right brace] (VF.north east);
        \draw [black, thick] (MF.south west) to [square left brace] (MF.north west);
        \draw [black, thick] (MF.south east) to [square right brace] (MF.north east);
        \draw [black, thick] (NF.south west) to [square left brace] (NF.north west);
        \draw [black, thick] (NF.south east) to [square right brace] (NF.north east);
        \draw [decorate, decoration={brace, raise=0.2cm, amplitude = 0.5cm}] (VF.north) to node[above=0.7cm] {分区结构(Feldermodell)} (NF.north);
        \draw [decorate, decoration={brace, raise=0.1cm, amplitude = 0.3cm, mirror}] (LSK.south) to node[below=0.4cm] {句框(Satzklammer)} (RSK.south);
    \end{scope}
\end{tikzpicture}
    % \caption{动词的分类}
    % \label{fig:verb-categories}
\end{figure}

大部分语言造句时的核心在于谓语,德语的谓语具有「分区结构」(Feldermodell),分区结构也和「句框」(Satzklammer)有紧密联系。

而句框的核心是将谓语分成「变位部分」与「非变位部分」,将变位部分放在句框首部(左框/LSK),非变位部分放在尾部(右框/RSK)。按照语言学的分析,分区结构也可以理解为动词第二顺位\footnotemark[1]。

\footnotetext[1]{
第二动词顺位主要和日尔曼语族语言有关,即不论何种情况,动词位于句子成分的第二位。动词第二顺位语言(V2语言)也可以分成两种主要的类型。CP-V2语言(如瑞典语、德语)只允许主要子句的动词移动。IP-V2语言(如冰岛语、意第绪语)则要求从属子句的动词也要移动。

有一种特定的语法理论认为有一个位置叫“补语标记”(complementiser),记为"C"。CP-V2语言里的从属子句里,本来动词应该要移位到第二顺位,但是因为C的位置已经被关系代名词(如英语的"that")占据了,从属子句的动词便不移动了。而在IP-V2语言里,紧接在C的后面还可以再放一个位置,记为"I"。在移动从属子句的动词之前,I的位置一定是空的,所以允许从属子句的动词能够移动到从属子句里的第二顺位。虽然解释这个现象的理论是依据前述的特定语法理论,然而瑞典语与德语、冰岛语与意第绪语在这个现象上彼此两两相似是事实,而一些即使不认同该特定语法理论的人也会用"CP-V2"、"IP-V2"之类的术语。

英语在较早期曾经是V2语言,一些当时的痕迹还可以在现在的英语里发现:像是"so am I"、"I didn't go and neither did he"等等。也有人认为更早的英语语序是主谓宾、IP-V2语言。}


需要按照句框构造的谓语可以有以下这些情况,加号左边为发生变位的部分(LSK),右边为不变位的部分(RSK):
\begin{itemize}
    \item「werden/haben/sein + 实义动词」
    \item 「情态动词/其他动词 + 不定式」
    \item 「词干 + 可分前缀」
    \item 「系动词 + 表语」
    \item 「功能动词 + 名词性部分」
    \item 「动词 + nicht」
\end{itemize}

\subsection{前区成分}
VF 通常包含主语,或表示时间/元音/状态/地点的说明语。如果句子的其他成分位于前场,则要么其是已知信息,要么为被强调的成分。

\subsubsection{通常的前区成分}

\begin{itemize}
    \item (形式)主语或说明语

    \example{\uline{Er} hat seinem Enkel das Buch am Freitag geschenkt.}{他周五送给孙子这本书。}
    \example{\uline{Mit diesen dreckigen Händen} kommst du nicht an den Tisch.}{你的手这么脏,不能上桌。}
    \example{\uline{Am Freitag} hat er seinem Enkel das Buch geschenkt.}{周五他把书送给你孙子。}
    \example{\uline{Es} freut mich, Sie kennen zu lernen.}{很高兴认识您。}
    \example{\uline{Es} wird nun gearbeitet.}{现在开始工作。}

    \item 特殊疑问词

    \example{\uline{Wem} gehört das Lehrbuch?}{这本教科书是谁的?}
    \example{\uline{Wie lange} sind Sie schon in China?}{您来到中国多久了?}
    \example{\uline{Wie} schmeckt dir der Kuchen.}{你觉得这蛋糕好吃吗?}

    \item zu + 不定式结构

    \example{\uline{Sich zu konzentrieren}, ist hier schwierig.}{在这里很难集中精力。}
    \example{\uline{Mit den Kindern umzugehen}, muss man noch richtig lernen.}{怎样跟孩子打交道,还得好好学学。}
    \example{\uline{Ohne sich zu verabschieden}, verließ er das Zimmer.}{他离开了房间,没有和别人告别。}

    \item 其他结构:不定式 / Ptzp.II / 功能动词中的名词性成分

    \example{\uline{Aufpassen} musst du.}{你得注意。}
    \example{\uline{Eine Entscheidung} ist sofort zu treffen.}{决定必须立即做出。}
    \example{\uline{In Kraft} wird das Gesetz erst ab 01.01.2010}{该法律于2010年1月1日才生效。}

    \item 某些具有独立意义的可分前缀

    \example{\uline{Hinzu} kommen noch die Lohnnebenkosten.}{还要加上额外工资。}
    \example{\uline{Hinauf} ist er über die Treppe gegangen.}{他经过楼梯走了上去。}
    \example{\uline{Herunter} ist sie von oben gekommen.}{她从上面下来。}

    \item 从句

    \example{\uline{Ob das stimmt}, weiß ich noch nicht.}{我不知道这是否正确。}
    \example{\uline{Weil er Fieber hat}, muss er im Bett bleiben.}{因为他发烧,必须卧床休息。}
    \example{\uline{Seit du gekommen bist}, hast du noch kein einziges Wort gesagt.}{从你来到这儿,你就一个字都没说过。}

    \item 前区为空:选择疑问句/命令式/非现实愿望句

    \example{\uline{\hspace*{2em}} Gehört dir das Lehrbuch?}{这本教科书是你的吗?}
    \example{\uline{\hspace*{2em}} Waren Sie schon mal in China?}{你以前到过中国吗?}
    \example{\uline{\hspace*{2em}} Geh zum Arzt!}{去看医生!}
    \example{\uline{\hspace*{2em}} Lass das Buch auf dem Tisch ligen!}{把书放桌上!}
    \example{\uline{\hspace*{2em}} Würde er doch endlich kommen!}{希望他终于来了!}
    \example{\uline{\hspace*{2em}} Hörte es doch endlich auf zu regnen!}{雨终于停了!}
\end{itemize}

\subsubsection{不可置于前区的成分}
\begin{itemize}
    \item 反身代词

    \sout{Sich hat er in sie verliebt.}

    \sout{Uns haben wir damit beschäftigt.}

    \item 无独立意义的可分前缀

    \sout{Zu stimme ich dir.}

    \sout{Ein lade ich euch herzlich zum Essen.}

    \item 不定冠词的 A、D 格

    \sout{Einen kann das schon ärgern.}

    \sout{Einem tut es leid.}

    \item 单独的否定词 nicht (也就是说 nicht + 被修饰成分可以放在前区)
    
    \sout{Nicht fährt der Zug nach Bern.}

    \sout{Nicht habe ich es verstanden.}

    \example{\uline{Nicht heute} findet die Veranstaltung statt.}{活动不在今天举行。}
    \example{\uline{Nicht am Freitag} haben wir die Prüfung.}{我们不是在周五有考试。}
    \example{\uline{Nicht er} hat es getan.}{不是他干的。}

\end{itemize}

\subsubsection{前区之前}
有一些词称为零占位词,它们放在前区之前,不占据前区的位置,通常为并列连词或者一些副词。如:aber, denn, oder, und, sondern; auch, besonders, fast, nur, sogar, selbst。

\example{Aber \uline{er} hat die Prüfung nicht bestanden.}{但是他没有通过考试。}
\example{\ldots, denn \uline{die Straßen} sind glatt.}{\ldots, 因为地滑。}
\example{Fast \uline{jeder Student} war mal in Deutschland.}{几乎每个学生都去过德国。}
\example{Sogar \uline{die teuereten} Uhren kann er sich leisten.}{甚至最贵的手表他都能买得起。}

但是有一些副词也具有连词功能,它们需要占据前区的位置。

\example{\uline{Deshalb} habe ich kein Interesse mehr daran.}{所以我对此就不再有兴趣了。}
\example{\uline{Seitdem} haben wir uns nicht gesehen.}{从那时起我们就没又再见面。}
\example{\uline{Trotzdem} ist er zur Arbeit gegangen.}{尽管如此,但他还是去上班了。}
\example{\uline{Daraufhin} rannte er aus dem Haus.}{于是他就跑出了房子。}

\subsection{中区成分与语序}
中区是句子信息最集中的地方,主语、宾语、补足语都可以放在中区,数量不限。关键之处则是由于各种成分都可以放,中区可能会过长,则说话时需要根据环境控制长度。而各成分的排列顺序由语法规则、语境、表达者意图共同决定的,没有刻板的规则,但是有通用的原则。通常来说,中区的前段是已知信息,后段是未知信息,人们习惯将要强调的部分放在句子的后段。

\subsubsection{中区语序}
通常来说,中区成分的排列按照以下顺序:主语 - 宾语[与格(D) - 宾格(A)] - 说明语[时间 - 原因 - 情状 - 地点] - 补足语 - 属格宾语(G)。有的观点认为中区只有受中心词(谓语)变位影响的「补足语」和不受变位影响的「说明语」两种成分,即主语、宾语是「主/与/宾格补足语」。总的来说,补足语是句子的必需成分,缺少会导致意思不完整,而说明与术语可有可无的成分。

则通常情况下,中区成分包括以下部分

\begin{longtblr}[
    theme=nocaption,
    presep={2pt},
    label = {tab:MF},
]{
    width=\linewidth,
    rowhead=1,
    rows={m,abovesep=0pt,belowsep=0pt},
    colspec={lXl},
    hline{3-10,12-19,21,22}={dotted},
    cell{2-4,10-12}{1-Z}={azure9},
    cell{5-9}{1-Z}={gray9},
}
    \textbf{类型} & \textbf{例句} & \textbf{疑问词} \\
    \hline
    {主格补足语(主语)\\Nominativ} & \uline{Er} versteht mich nicht. & Wer? / Was? \\
    {与格补足语\\Dativ} & Das Buch gehört \uline{meinem} Freund. & Wem? \\
    {宾格补足语\\Akkusativ} & Ich verstehe \uline{ihn} gut. & Wen? / Was? \\
    {时间说明语\\Temporal} & Er kommt \uline{morgen} zurück. & Wann? \\
    {原因说明语\\Kausal} & \uline{Er} \uline{Wegen des Gewitters} blieb ich im Haus. & Warum? \\
    {情状说明语\\Modal} & Otto musste \uline{schwer} arbeiten. & \\
    {地点说明语\\Lokal} & Ich habe ihn \uline{am Bahnhof} getorffen. & Wo? \\
    {意图说明语\\Final} & Ich fahre \uline{zur Erholung} an die See  & Wozu? \\
    \SetCell[r=2]{m} {介词补足语\\Präpositional} & Ich interessiere mich \uline{für alte Musik}. & Wofür? / Für wen? \\
    & \uline{Auf Peter} können wir nicht warten. & Woauf? / Auf wen? \\
    {属格补足语\\Genitiv} & Der Kranke bedarf \uline{ärztlicher Hilfe}. & Wessen? \\
    {条件说明语\\Konditional} & \uline{Bei diesem Lärm} kann ich nicht leanen. & \\
    {让步说明语\\Konzessiv} & \uline{Trotz} des Regens gehen wir spazieren. & \\
    {方式说明语\\Instrumental} & Wir waschen uns die Hände \uline{mit Seife}. & \\
    {引用说明语\\Referenz} & \uline{Meiner Meinung nach} ist das zu schwer. & \\
    {否定说明语\\Negations} & Petra kommt heute \uline{nicht}. & \\
    {地点补足语\\Situativ} & Die Insel Rügen liegt in der \uline{Ostsee}. & Wo? \\
    \SetCell[r=2]{m} {方向补足语\\Direktiv} & Ich stecke den Brief \uline{in die Tasche.} & Wohin? \\
    & Herr Kim stammt \uline{aus Korea}. & Woher? \\
    {数量补足语\\Expansiv} & Der Eintritt kostet \uline{einen Euro}. & Wie Viel? \\
    \SetCell[r=2]{m}{名词性补足语\\Nominal} & Margret ist \uline{Lehrerin}. & Was? \\
    & Wir betrachten \uline{als unseren Freund}. & Als Was? \\
\end{longtblr}

\subsubsection{小品词}
如果一句话出现几个小品词,它们的顺序是有一定规则的:
\begin{itemize}
    \item 陈述句:ja - denn - eben/halt - doch - wohl - einfach - schon - auch - mal
    \item 疑问句:denn - wohl - etwa - schon - auch - nur - bloß
    \item 命令式:doch - eben - einfach - schon -auch - nur - bloß - ruhig - mal - ja
\end{itemize}

\subsubsection{中区语序的几个原则}
中区成分除了受到语法的影响,还有以下三个原则
\begin{itemize}
    \item 尾部焦点原则(alt vor neu):即已知信息偏向句子前部,未知信息偏向句子后部

    \example{-Warum wurde er bezichitigt?\\-Man bezichtigte \ult{ihn gestern vor Gericht}{alt} \ult{aus unerfindlichen Gründen des Diebstahls}{neu}.}{-他为什么被指控?\\-他昨天在法庭上出于不明原因被指控偷窃。}

    \example{-Sie hat eine Halskette gekauft.\\-Sie schenkte \ult{die Halskette}{alt} \ult{ihrer Freundin}{neu}.}{-她买了一条项链。\\-她把项链送给了朋友。}

    \item 代词优先原则(pronomen vor nomen):当主/宾/与格补足语中同时有代词与名词时,代词(不定/人称/反身)总是位于名词之前。如果代词位于名词后,一般是出于强调该代词的意图,口语中也会重读该词。

    \example{Der Großvater hat \ult{es}{代词} \ult{seinem Enkel}{名词} geschenkt.}{祖父把它送给了孙子。}

    \example{Er hat \ult{sie}{代词} \ult{seinen Kommilitonen}{名词} vorgestellt.}{他把她介绍给同学们。}

    \item 生命优先原则(belebt vor unbelebt):除了代词,有生命的名词成分一般位于无生命的之前,主要适用于一些支配与格宾语的动词 prssieren, fehlen, einfallen, gefallen, misslingen, stehen, schmecken 等。
    
    \example{Dafür fehlte \ult{de Leuten}{belebt} \ult{das Geld}{unbelebt}.}{为此人们还缺少钱。}

    \example{Seine Ungeschicktheit hat \ult{den Angestellten}{belebt} \ult{die Stelle}{unbelebt} gekostet.}{他的不小心使这位职员丢掉了工作。}
\end{itemize}

\subsection{破框和后区成分}

句框结构这一规则导致的后果就是,在有很长的从句或很多限定词时导致 MF 太长,需要等到 MF 结束后才能知道 RSK 究竟是什么,因此在一些情况下,MF中的一些成分可以提取出来,放到 NF。这种将成分从 MF 移到 NF 的情况称为「破框」(Ausklammerung),取「打破句框」之意,实则是将一些成分放至 NF。这并不是一种「固定搭配」「经验规则」,而是合规的语法变化。这种行为常见于口语,目的使对话更易理解。

\subsubsection{通常的后区成分}\label{sec:normal_NF}
\begin{itemize}
    \item zu + 不定式: 若不定式结构简单,可放于 MF
    
    \example{Er hat mir geholfen, \uline{einen Absatzplan für das nächste Quartal zu erstellen}.}{他帮我制定下个季度的销售计划。}

    \example{Du hast mich doch versprochen, \uline{mich morgen um halb acht abzuholen}.}{你可是答应过我明天七点半去接我。}

    \item 带有 wie ... als ... 的比较说明语: 若 als 不表示比较,而表示「作为」时,通常需要位于中场,但若 als 后结构复杂,还是可以置于 NF。
    
    \example{Er ist heute schneller gerannt, \uline{als im gestrigen Rennen}}{他今天比昨天跑得要快。}

    \example{Er is heute schneller \uline{als im gestrigen Rennen} gerannt.}{他今天跑得比昨天要快。}

    \example{Wir haben ihn \uline{als aufrichtigen Menschen kennen} gelernt.}{我认识的他是一个正直的人}

    \example{Wir haben ihn kennen gelernt \uline{als aufrichtigen Menschen, guten Freund, ehrlichen Kollegen, und engargierten Gemeindebürger.}}{我认识的他是一个正直的人、一个好朋友、真诚的同事和热心的市民。}

    \item 介词词组: 虽然介词词组也可置于 MF,但现代德语有将其后置的趋势,尤其是当结构较为复杂时。同时,也可起到表达强调语气的作用,如同添加了 und zwar。但如果介词词组不完整,如代/副词形式的介词不可位于 NF。

    \example{Die Aufgabe wurde gelöst \uline{durch eine erfahrene, aus Vertretern verschiedener deutscher Universitäten und technischer Hochschulen zusammengesetzte Expertengruppe}.}{这项任务是由一支有经验的,由多家德国综合性和技术高效代表组成的专家小组完成的。}

    \example{Warum sollte ich dann noch reden, (und zwar) \uline{mit diesem unausstehlichen Menschen!}}{我为什么还要讲话,而且是跟这么一个令人讨厌的人!}

    \example{\sout{Ich hab nicht gedach daran.}\\Ich hab nitch daran gedacht.}{这我没有想到。}

    \example{\sout{Wir haben uns entschieden dafür}, ein Vertretungsbüro in China aufzustellen.\\Wir haben uns dafür entschieden, ein Vertretungsbüro in China aufzustellen.}{我们决定在中国设立一个代表处。}

    \item 同位语: 如果同位语的修饰成分不是过于紧密,可以位于 NF, 或为了取得特殊的文体效果。

    \example{Der Redner hat auch Helmut Schmidt gewürdigt, \uline{den international bekannten Politiker und in Deutschland immer noch von vielen geschätzten Albundeskanzler}.}{演讲者也赞扬了 H·S, 这位世界知名的政治家,在德国至今受到广泛尊敬的老德国总理。}

    \example{Er ist an Krebs gestorben, \uline{dieser heimtückischen Krankheit}.}{他死于癌症,这种恶性疾病。}

    \item 从句: 大多数情况从句位于 NF,但有时也可位于 MF
    
    \example{Kannst du mir sagen, \uline{wie spät es ist}?}{你能告诉我现在位于几点了吗?}

    \example{Wir sind, \uline{weil es so stark regnete}, nicht ans Meer gegangen.}{由于雨下的很大,我们没有去海边。}
\end{itemize}

\subsubsection{不可放置于后区的成分}

\begin{itemize}
    \item 主语、宾语

    \example{\sout{Leider ist immer noch nicht eingetroffen der Auftrag.}}{可惜订单一直没到。}

    \item 表语

    \example{\sout{Sie wollte schon immer werden Pilotin bei der Luftwaffe.}}{她一直想成为空军飞行员。}

    \item 无介词引导的状语

    \example{\sout{Du darfst nicht reden so unanständig.}}{你讲话不能这么不正经。}

    \item 代副词

    \example{\sout{Er hat sich bedankt dafür, dass Sie ihm in solcheiner schwierigen Lage beistehen.}}{他感谢您在这样的困境下还给他提供帮助。}

    \item 小品词

    \example{\sout{das hättest du nicht machen sollen aber.}}{你可不应该这样做。}

\end{itemize}
