\section{动词与变位}

\subsection{T-A-M-V 系统}

屈折语大都具有 T-A-M 或 T-A-M-V 系统,即 Tense-Aspect-Mood-Voice (时-体-式(气)-态),如过去时-完成体-虚拟式-被动态的组合,它们的默认值通常是现在时-一般体-陈述式-主动态。TAMV 系统主要应用于句子中谓语部分的变化。

德语也有类似的系统:
\begin{enumerate}[leftmargin=3.5em, topsep=0pt, itemsep=0pt, parsep=0pt]
    \item Tempus(Tense): Präsens (现在时),Präteritum (过去时),Futur (将来时)
    \item Aspekt(Aspect): Simple (一般体),Perfekt (完成体)
    \item Modus(Mood): Indikativ (直陈式),Konjunktiv I (第一虚拟式),Konjunktiv II (第二虚拟式),Imperativ (命令式)
    \item Diathese/Aktionsform(Voice): Aktiv (主动态),Passiv (被动态)
\end{enumerate}

因此,理论上应该有 $3*2*4*2=48$ 种时-体-式-态的变化,但是实际上很多并不存在。如果不考虑被动态,有$15$种时-体-式的组合。但关注到动词上,德语动词的变位除了 TAMV 系统之外,还受到人称的影响。

这 $15$ 种组合如\cref{tab:basic-tense-mood}。按照实际教学中的习惯,将「Tempus」和「Aspekt」的六种组合统称为「Tempus」,其中第一将来时与第二将来时可以分别看作「一般将来时」「将来完成时」。

\begin{table}[htbp]
    \caption{直陈式下的时-体-式变化}
    \label{tab:basic-tense-mood}
    \centering
    \footnotesize
\begin{tblr}{
    width=\textwidth,
    vlines,
    hlines,
    cell{3}{3}={r=3}{m,l},
    cell{3}{4}={r=3}{m,l},
    cell{3}{Z}={r=5}{m,c},
    rows={m},
    column{2-Z}={colsep=1.5pt},
}
    \diagbox{Tempus}{Modus} & {直陈式\\(Indikativ)} & {第一虚拟式\\(Konjunktiv I)} & {第二虚拟式\\(Konjunktiv II)} & {命令式\\(Imperativ)} \\
    {现在时\\(Präsens)} & Er \e{tut} es & Er \e{tue} es & Er \e[markgreen]{würde} es \e{tun} & \e{Tu} es! \\
    {过去时\\(Präteritum)} & Er \e{tat} es & {Er \e[markgreen]{habe} es \e{getan}} & {Er \e[markgreen]{hätte} es \e{getan}} &  \\
    {完成时\\(Perfekt)} & Er \e[markgreen]{hat} es \e{getan} &       &       &  \\
    {过去完成时\\(Plusquamperfekt)} & Er \e[markgreen]{hatte} es \e{getan} &       &       &  \\
    {第一将来时\\(Futur I)} & Er \e[markgreen]{wird} es \e{tun} & Er \e[markgreen]{werde} es \e{tun} & Er \e[markgreen]{würde} es \e{tun} &  \\
    {第二将来时\\(Futur II)} & Er \e[markgreen]{wird} es \e{getan} \e[markgreen]{haben} & Er \e[markgreen]{werde} es \e{getan} \e[markgreen]{haben} & Er \e[markgreen]{würde} es \e{getan} \e[markgreen]{haben} &  \\
\end{tblr}
\end{table}


表中绿色标识了助动词,黄色标识了实义动词「tun」,表示做、干。可以发现这些时态似乎能与英语的时态有较多对应,但在使用上并不完全一致,尤其是一般过去时与完成式上,且将来时的第二虚拟式变化较少使用。

对 Modus 简单进行介绍:
\begin{enumerate}[leftmargin=3.5em, topsep=0pt, itemsep=0pt, parsep=0pt]
    \item 直陈式: 最常用的 Mood, 常用于陈述事实
    \item 第一虚拟式: 也称为「简介虚拟式」,常用于转述他人的话,常见于新闻或报纸中
    \item 第二虚拟式: 也称为「非现实虚拟式」,类似英语中的虚拟语气,表达一种假设
    \item 命令式: 类似英语中的祈使句,表达命令
\end{enumerate}

对 Tempus 简单进行介绍:
\begin{enumerate}[leftmargin=3.5em, topsep=0pt, itemsep=0pt, parsep=0pt]
    \item 现在时: 类比于英语的一般现在时,也可以表达强调的语气(I {\bf do} take the bus),或表示未来(I get paid tomorrow),或表达正在进行的状态(I visit my parents this weekend)。
    \item 过去时、完成时: 德语口语中更偏向与使用完成时替代过去时
    \item 过去完成时: 过去时+完成体
    \item 第一将来时: 将来时+一般体
    \item 第二将来时: 将来时+完成体
\end{enumerate}

\subsection{动词的分类}

如\cref{fig:verb-categories},德语的动词可以从多种维度进行分类。

\begin{figure}[H]
    \centering
    \begin{tikzpicture}[
    level distance=1.5cm,
    level 2/.style={level distance=2.2cm},
    level 3/.style={level distance=2cm},
    edge from parent path={(\tikzparentnode.south) -- +(0,-12pt) -| (\tikzchildnode)},
    every tree node/.style={align=center,anchor=north},
    ]

    \begin{scope}
        \Tree
        [.\node[draw]{用途};
            [.{情态动词} ]
            [.{助动词} ]
            [.{实义动词} ] ]
    \end{scope}

    \begin{scope}[xshift=5cm]
        \Tree
        [.\node[draw]{宾语};
            [.{及物动词} ]
            [.{不及物动词} ] ]
    \end{scope}

    \begin{scope}[yshift=-2.7cm]
        \Tree
        [.\node[draw]{分词搭配助动词};
            [.{使用haben} ]
            [.{使用sein} ] ]
    \end{scope}

    \begin{scope}[xshift=5cm, yshift=-2.7cm]
        \Tree
        [.\node[draw]{其它};
            [.{反身动词} ]
            [.{可分/不可分动词} ] ]
    \end{scope}

    \begin{scope}[xshift=2.5cm, yshift=-5.4cm]
        \Tree
        [.\node[draw]{变位};
            [.{不规则动词\\unregelmäßige \boxed{unr.}}
                [.{强变化动词\\starke}
                    [.{元音音变规律\\A/B/C} ] ]
                [.{混合变化动词\\gemischte} ] ]
            [.{规则动词\\regelmäßige}
                [.{弱变化动词\\schwache \boxed{sw.}} ] ] ]
    \end{scope}
\end{tikzpicture}
    \caption{动词的分类}
    \label{fig:verb-categories}
\end{figure}

\subsubsection{动词的前缀}

同英语一样,通过添加前缀可以为单词赋予新的含义(do, redo, undo),但德语中这一现象更普遍且稍有不同。

德语动词的前缀可以分为两类:可分前缀与不可分前缀。不可分前缀更像英语中单词的前缀({\bfseries be}kommen <=> {\bfseries be}come),但也不是一一对应的,拼读时不可分前缀通常轻读。而可分前缀则更像动词短语中的介词({\bfseries zu}nehmen <=> go {\bfseries up}),在动词发生变位时将被置于句尾,拼读时前缀需要重读。

同时,对于同一个动词,一个前缀即可以是可分前缀也可以是不可分的,主要通过读音区分,它们在含义上的区别大致可以体现为:含义若更贴近字面语言(literal),则是不可分前缀,更贴近比喻语言(figurative)的则是可分前缀。如 {\bf durch}suchen,前缀 durch 表示 through,词根 suchen 表示 search。若当作不可分前缀,则轻读 durch,表示「完整的搜索(一个物理空间)」,如执法人员在搜索现场。若当作可分前缀,表示「在(电子文档、邮件、记录等)中搜索」。

\subsubsection{动词的宾语}

德语中有一些动词在表达某一含义时,必须搭配反身代词,这些动词也称为反身动词,在表示时通常使用第三人称的反身代词「sich+不定式」,词典中词条前的 sich 也表示该义项是反身代词的用法,也建议记忆时记忆「sich+不定式」以避免混淆。

德语中有一些动词在表达某一含义时,必须搭配与格(第三格/Dative)名词,即没有直接宾语但依旧要使用与格,这样的单词大约有 $50$ 个左右。


\subsubsection{动词的变位与用途}

按照动词的变位规律,可以将其分成三类或两类,即「强变化/混合变化/弱变化动词」或「不规则/规则动词」,三分的方法常见于较旧的教学中,新的教学中主要使用二分的方法,也就是将强/混合变化统称为不规则动词。

德语动词的变位主要体现在三个地方:动词的前缀、词干元音音变、词尾变化。弱变化动词的变化均遵循同一套规律,因此只需记忆规律则可自行推导出动词的相关形式。而不规则动词并没有统一的变化规律,需要一个个单独记忆。德语中需要常用的不规则动词在 $220$ 个左右,其它几千个动词都是弱变化动词,同时不规则动词又可以依据其元音音变的规律分成三类,能进一步地减轻记忆压力。


除了实义动词,德语中有 $6$ 个情态动词与 $3$ 个助动词,它们都是不规则动词,并且用于构成不同的 T-M-A-V 系统。


\subsection{动词可能的变化}

对于一个动词来说,动词的不定式(Infinitiv)是基本的形式,也是出现在词典上的形式,也可以被称为「动词原型」。动词还具有现在分词、过去分词两个分词形式,以及动名词形式。在不定式的基础上,动词的变位受到人称、数、时、体、式的影响,有直陈式下的十二种基础变位形式,第一、第二虚拟式与命令式的其他变位形式。这些变位都有统一的规律,都基于十二种基础变位,并且大部分动词都完全按照这些规律变化,所以只需要记住基本的规律就好。

总地来说,关于动词的变位,建议首先要理解相关概念后,再按照以下顺序的优先级记忆:

\begin{enumerate}[leftmargin=3.5em, topsep=0pt, itemsep=0pt, parsep=0pt]
    \item 直陈式下的变位规律(十二个)
    \item 构成过去分词的规律
    \item 助动词、情态动词的所有变位(包括命令、虚拟式)
    \item 常见的动词前缀
    \item 常用不规则动词的「现在时第三人称单数、过去时第三人称单数、过去分词」
    \item 构成现在分词与动名词的规律
    \item 构成命令式的变位规律
    \item 少数动词的反身用法
    \item 少数动词的与格用法
    \item 构成虚拟式的变位规律
\end{enumerate}

这里的前五条是重点,尤其是助动词、情态动词的所有变位,这是由于各种 T-A-M-V 的变化主要依靠助动词、情态动词的变位来体现。

\subsubsection{动词的基础变位}

在直陈式下,动词有 $2*6=12$ 种变格形式,即两种时和六种人称的组合,这是动词的基本变格形式。在构成虚拟式和命令式时,会有新的变格规则,但对绝大部分动词是一样的。只有在构成第二虚拟式时,但一些特定的单词(不规则动词中的不规则动词),会有不规则的变格。但在实际构成第二虚拟式时,只有助动词和情态动词按照第二虚拟式要求变格,因此不用担心,只需要记住助动词和情态动词的变格形式即可。

不定式是动词的基本形式,也可以叫做动词原型,在德语里以「-en」结尾。若去掉不定式的词尾「-en」即是动词的词干,是表达动词词义的主要部分,也是在变格时不发生变化的部分(如果不考虑元音音变)。但对于以「t/d」结尾的词干,为了发音方便,在一些变格时,将不定式去掉「-n」后作为动词词干,即保留结尾的「e」。

如\cref{tab:weak-indikativ-conjugation},标黄的部分是变位时发生变化的词尾,词干部分不变。但是需要注意的是,在词干以「t/d」结尾时的单数第一、复数一、三人称时,虽然词干保留了 e,但词尾又以 e 开头,连写 e 会改变元音的读法,因此将词干和词尾的 e 合为一个。在其他式的变格,如果遇到了相似的问题,还是一样的处理方式。中心思想就是:为了发音方便保留 e,为了不改变元音发音合并 e。

\begin{table}[htbp]
    \caption{弱变化动词的基本变位}
    \label{tab:weak-indikativ-conjugation}
    \centering
\begin{tblr}{
    width=\textwidth,
    cell{2}{1}={r=2}{m,c},
    columns={l,colsep=5pt},
    hline{even[1-Y]}={},
    vline{2,3}={},
}
    & & ich   & du    & er/sie/es & wir   & ihr   & sie/Sie \\
    弱变化 & Präs. & -e    & -st   & -t    & -en   & -t    & -en \\
    & Prät. & -te   & -test & -te   & -ten  & -tet  & -ten \\
    kaufen & Präs. & kauf\e{e} & kauf\e{st} & kauf\e{t~} & kauf\e{en} & kauf\e{t~} & kauf\e{en} \\
    kauf- & Prät. & kauf\e{te} & kauf\e{test} & kauf\e{te} & kauf\e{ten} & kauf\e{tet} & kauf\e{ten} \\
    reden & Präs. & red\e{(e)} & rede\e{st} & rede\e{t~} & red\e{(e)n} & rede\e{t~} & red\e{(e)n} \\
    red(e)- & Prät. & rede\e{te} & rede\e{test} & rede\e{te} & rede\e{ten} & rede\e{tet} & rede\e{ten} \\
    arbeiten & Präs. & arbeit\e{(e)} & arbeite\e{st} & arbeite\e{t~} & arbeit\e{(e)n} & arbeite\e{t~} & arbeit\e{(e)n} \\
    arbeit(e) & Prät. & arbeite\e{te} & arbeite\e{test} & arbeite\e{te} & arbeite\e{ten} & arbeite\e{tet} & arbeite\e{ten} \\
\end{tblr}
\end{table}


强变化动词则稍微复杂。总的来说,强变化动词除了词尾变化,还有词干的元音音变。

强变化动词构成现在时,大部分动词不发生音变,但有少部分动词的单数第二、三人称会发生元音音变;构成过去时,则所有人称的词干都发生一样的音变。

强变化动词构成现在时,词尾要按照弱变化动词的现在时词尾变位;构成过去时,单数第二人称与复数所有人称词尾与对应人称的单数词尾相同。

以\cref{tab:strong-indikativ-conjugation}为例,词尾的变化依旧通过黄色标识,绿色底色标识现在时变位时词干可能发生音变,红色底色标识过去时变位词干发生音变。

\begin{table}[htbp]
    \caption{强变化动词的基本变位}
    \label{tab:strong-indikativ-conjugation}
    \centering
\begin{tblr}{
    width=\textwidth,
    cell{2}{1}={r=2}{m,c},
    cell{2}{4,5}={markgreen},
    cell{3}{3-Z}={markred},
    columns={l},
    hline{even[1-Y]}={},
    vline{2,3}={},
}
        & & ich   & du    & er/sie/es & wir   & ihr   & sie/Sie \\
        强变化 & Präs. & -e    & -st   & -t    & -en   & -t    & -en \\
        & Prät. &    & \boxtbl{1.4em}{2.3em} -st &   & \boxtbl{1.4em}{2.3em} -en  & \boxtbl{1em}{2.3em} -t  & \boxtbl{1.4em}{2.3em} -en \\
        trinken & Präs. & trink\e{e} & trink\e{st} & trink\e{t~} & trink\e{en} & trink\e{t~} & trink\e{en} \\
        trink- & Prät. &  \e[markred]{trank} & \e[markred]{trank}\e{st} & \e[markred]{trank} & \e[markred]{trank}\e{en} & \e[markred]{trank}\e{t} & \e[markred]{trank}\e{en} \\
        fallen & Präs. & fall\e{(e)} & \e[markgreen]{fäll}\e{st} & \e[markgreen]{fäll}\e{t~} & fall\e{(e)n} & fall\e{t~} & fall\e{(e)n} \\
        fall- & Prät. & \e[markred]{fiel} & \e[markred]{fiel}\e{st} & \e[markred]{fiel} & \e[markred]{fiel}\e{en} & \e[markred]{fiel}\e{t} & \e[markred]{fiel}\e{en} \\
\end{tblr}
\end{table}

对于混合变化动词,如\cref{tab:mixed-indikativ-conjugation},如在现在时变位时,与弱变化相同,同时不会出现像强变化动词那样的元音音变;在过去时变位时,词干需要发生元音音变,同时词尾按照弱变化动词词尾变化,与强变化动词不同。

\begin{table}[htbp]
    \caption{混合变化动词的基本变位}
    \label{tab:mixed-indikativ-conjugation}
    \centering
\begin{tblr}{
    width=\textwidth,
    cell{2}{1}={r=2}{m,c},
    cell{3}{3-Z}={markred},
    columns={l},
    hline{even[1-Y]}={},
    vline{2,3}={},
}
        & & ich   & du    & er/sie/es & wir   & ihr   & sie/Sie \\
        混合变化 & Präs. & -e    &  -st   &  -t    & -en   & -t    & -en \\
        & Prät. &  -te    &  -test &   -te &  -ten  &  -tet  &  -ten \\
        bringen & Präs. & bring\e{e} & bring\e{st} & bring\e{t~} & bring\e{en} & bring\e{t~} & bring\e{en} \\
        bring- & Prät. &  \e[markred]{brach}\e{te} & \e[markred]{brach}\e{test} & \e[markred]{brach}\e{te} & \e[markred]{brach}\e{ten} & \e[markred]{brach}\e{tet} & \e[markred]{brach}\e{ten} \\
        wissen & Präs. & \e[markgreen]{weiß} & \e[markgreen]{weißt} & \e[markgreen]{weiß} & wiss\e{en} & wiss\e{t~} & wiss\e{en} \\
        wiss- & Prät. &  \e[markred]{wuss}\e{te} & \e[markred]{wuss}\e{test} & \e[markred]{wuss}\e{te} & \e[markred]{wuss}\e{ten} & \e[markred]{wuss}\e{tet} & \e[markred]{wuss}\e{ten} \\
\end{tblr}
\end{table}
需要注意的是,混合变化动词中只有「wissen」有不规则的变化,其他都没有。

\subsubsection{动词的分词与动名词}

动词可以通过变化成其现在分词(Partizip Präsens),也称为第一分词(Partizip I.),使其变成形容词,与英语中表达正在进行的事情的含义类似。值得一提的是,动词的过去分词也可以当作形容词使用,只不过含义与现在分词不同。此时按照形容词变格形式变化。德语所有动词构成现在分词的规则是一样的,使用「不定式+d」作为形容词词干,并搭配不同的形容词词尾。如 fließen => fließen-d-er。


为了构成完成体,动词需要变为其过去分词(Partizip Perfekt),也称为第二分词(Partizip II.),动词的变形遵循以下规律:
\begin{enumerate}[leftmargin=3.5em, topsep=0pt, itemsep=0pt, parsep=0pt]
    \item 默认使用「ge-词干-t」构成过去分词
    \item 如果动词以「-ieren」结尾,则不加「ge-」
    \item 如果动词含有不可分前缀(untrennbare Vorsilben)/轻读前缀,则不加「ge-」
    \item 如果动词具有可分前缀(trennbare Vorsilben)/重读前缀,则「-ge-」加在可分前缀与词干之间
    \item 大部分强变化动词添加「ge-」前缀,词干发生元音音变,但无词尾变化
\end{enumerate}
弱变化动词完全遵循此规律,不规则动词则词干要发生音变,也就是要单独记忆。需要注意的是,助动词 werden 有两个过去分词 geworden 和 worden,具体使用需要根据上下文选择。同时很多以 ieren 结尾的单词词干与英文中具有类似含义的单词类似,但这些单词的词源大都是法语词或德语词。

动词也可以通过变化成为动名词(Gerundium),即动词当作名词使用。所有动词构成动名词的规则也是一样的,即大写不定式的首字母。这样得到的名词均为中性名词,在变为复数时无元音音变或词尾变化。如 rauchen => das Rauchen。

\subsubsection{动词的虚拟式与命令式}

为了构成第一虚拟式,动词的变形遵循以下规律:
\begin{enumerate}[leftmargin=3.5em, topsep=0pt, itemsep=0pt, parsep=0pt]
    \item 在单数人称,复数第二人称时,使用「词干+e」,其它人称使用不定式
    \item sein 作为例外有单独的变化形式
\end{enumerate}
这种变形非常简单,但可以发现很多变位形式与现在时-一般时下的变位一样,因此在使用时,更偏向于使用第二虚拟式以避免混淆,或者在新闻/报纸中使用第三人称的第一虚拟式。

为了构成第二虚拟式,动词的变形遵循以下规律:
\begin{enumerate}[leftmargin=3.5em, topsep=0pt, itemsep=0pt, parsep=0pt]
    \item 弱变化动词使用{\bf 一般体-过去时}的变位
    \item 不规则动词基于{\bf 一般体-过去时}的词干,词干元音添加变音符号,并像第一虚拟式一样,在单数人称,复数第二人称时词干添加「-e」
    \item 部分强变化动词,词干元音音变不是添加变音符号,而是使用「ü」替换原有元音
    \item sollen 和 wollen 不发生变音
\end{enumerate}
似乎看起来很复杂,但不必太难过。在实际使用中,构成第二虚拟式时,助动词与情态动词总要变位,且绝大部分情况下其他动词不必变化。因此助动词和情态动词的第二虚拟式变位必需记住。

为了构成命令式,动词的变形遵循以下规律:
\begin{enumerate}[leftmargin=3.5em, topsep=0pt, itemsep=0pt, parsep=0pt]
    \item 使用非正式单数第二人称(du),使用一般体-现在时的 du 变位,并去掉词尾「-st」
    \item 使用非正式复数第二人称(ihr),使用一般体-现在时的 du 变位
    \item 使用正式第二人称(Sie),与上一条一样,使用一般体-现在时的 du 变位,但 Sie 需要添加在动词之后,同时为了缓和语气,通常还会添加 bitte
\end{enumerate}
命令式除了上述规律,还有「不定式命令式」与「建议型命令式」,在后续说明。


\subsection{动词变化的总结}

如\cref{tab:indikativ-conjugation},三种不同的底色代表词干发生不同的元音音变,也是不规则动词记忆的主要难点。同时增加了 sein, sollen 和 wollen 在虚拟式的不规则变位,其他变位都遵循规则。


\begin{table}[htbp]
    \caption{基础变位表}
    \label{tab:indikativ-conjugation}
    \centering
\begin{tblr}{
    width=\textwidth,
    vline{3,4,Y},
    hline{2,4,5,7},hline{even[10-Y]},
    column{1}={wd=1em},
    columns={l,colsep=2.3pt},
    rows={m},
    cell{2,5}{1}={r=2,c=2}{},
    cell{2,5}{Z}={r=2}{},
    cell{4}{1}={r=1,c=2}{},
    cell{4,6}{4-Y}={markred},
    cell{5}{5,6}={markgreen},
    cell{5}{Z}={markyellow},
    cell{7}{1}={r=7}{},
    cell{7}{2,Z}={r=3}{},
    cell{14}{1}={r=12}{},
    cell{even[10-Y]}{2,Z}={r=2}{},
    % cell{26}{1}={r=18}{},
}
    & & & ich   & du    & er/sie/es & wir   & ihr   & sie/Sie & Ptzp.II \\
    弱变化 & & Präs. & -e    &  -st   &  -t  & -en   & -t    & -en & ge+St.+t \\
    & & Prät. &  -te    &  -test &   -te &  -ten  &  -tet  &  -ten & \\
    混合变化 & & Prät. &  -te    &  -test &   -te &  -ten  &  -tet  &  -ten & \\
    强变化 & & Präs. & -e    &  -st   &  -t  & -en   & -t    & -en & ge+St.+en \\
    & & Prät. &     &  \boxtbl{1.4em}{2.3em} -st &    &  \boxtbl{1.4em}{2.3em} -en  & \boxtbl{1em}{2.3em} -t  &  \boxtbl{1.4em}{2.3em} -en & \\
    助动词 & sein & Präs. & bin   & bist  & ist   & sind  & seid  & sind & gewesen \\
    &       & Prät.   & war   & warst & war   & waren & wart  & waren & \\
    &       & Konj.I  & sei   & seist & sei   & seien & seit  & seien & \\
    & haben & Präs. & habe  & hast  & hat   & haben & habt  & haben & gehabt \\
    &       & Prät.   & hatte & hattest & hatte & hatten & hattet & hatten & \\
    & werden & Präs. & werde & wirst & wird  & werden & werdet & werden & (ge)worden \\
    &       & Prät.   & wurde & wurdest & wurde & wurden & wurdet & wurden & \\
    情态动词 & dürfen & Präs  & darf  & darfst & darf  & dürfen & dürft & dürfen & gedurft  \\
    &       & Prät  & durfte & durftest & durfte & durften & durftet & durften & \\
    & können & Präs  & kann  & kannst & kann  & können & könnt & können & gekonnt  \\
    &       & Prät  & konnte & konntest & konnte & konnten & konntet & konnten & \\
    & mögen & Präs  & mag   & magst & mag   & mögen & mögt  & mögen & gemocht  \\
    &       & Prät  & mochte & mochtest & mochte & mochten & mochtet & mochten & \\
    & müssen & Präs  & muss  & musst & muss  & müssen & müsst & müssen & gemusst  \\
    &       & Prät  & musste & musstest & musste & mussten & musstet & mussten & \\
    & sollen & Präs  & soll  & sollst & soll  & sollen & sollt & sollen & gesollt  \\
    &       & {Prät\\Konj.II}  & sollte & solltest & sollte & sollten & solltet & sollten & \\
    & wollen & Präs  & will  & willst & will  & wollen & wollt & wollen & gewollt  \\
    &       & {Prät\\Konj.II}  & wollte & wolltest & wollte & wollten & wolltet & wollten & \\
    % 混合变化动词 & brennen & Präs. & brenne & brennst & brennt & brennen & brennt & brennen & \\
    % &       & Prät. & brannte & branntest & brannte & brannten & branntet & brannten & \\
    % & bringen & Präs. & bringe & bringst & bringt & bringen & bringt & bringen & \\
    % &       & Prät. & brachte & brachtest & brachte & brachten & brachtet & brachten & \\
    % & denken & Präs. & denke & denkst & denkt & denken & denkt & denken & \\
    % &       & Prät. & dachte & dachtest & dachte & dachten & dachtet & dachten & \\
    % & kennen & Präs  & kenne & kennst & kennt & kennen & kennt & kennen & \\
    % &       & Prät  & keannte & kanntest & kannte & kannten & kanntet & kannten & \\
    % & nennen & Präs  & nenne & nennst & nennt & nennen & nennt & nennen & \\
    % &       & Prät  & neannte & nanntest & nannte & nannten & nanntet & nannten & \\
    % & rennen & Präs  & renne & rennst & rennt & rennen & rennt & rennen & \\
    % &       & Prät  & reannte & ranntest & rannte & rannten & ranntet & rannten & \\
    % & senden & Präs  & sende & sendest & sendet & senden & sendet & senden & \\
    % &       & Prät  & sandte & sandtest & sandte & sandten & sandtet & sandten & \\
    % & wenden & Präs  & wende & wendest & wendt & wenden & wendt & wenden & \\
    % &       & Prät  & wendete & wendetest & wendete & wendeten & wendetet & wendeten & \\
    % & wissen & Präs  & weiß  & weißt & weiß  & wissen & wisst & wissen & \\
    % &       & Prät  & wusste & wusstest & wusste & wussten & wusstet & wussten & \\
\end{tblr}
\end{table}

\subsection{直陈式}

\subsubsection{现在时}

现在时使用「动词的现在时变位」

\subsubsection{过去时与完成体}

过去时使用「动词的过去时变位」

完成体使用 「haben现在时变位 + 过去分词」

过去完成体使用 「haben过去时变位 + 过去分词」

\subsubsection{将来时}

第一将来时使用 「werden现在时变位 + 不定式」

第二将来时使用 「werden现在时变位 + 过去分词 + (搭配实义动词的)haben/sein」

\subsection{虚拟式}

若使用虚拟式,过去时、完成时、过去完成时均表达为过去时。

\subsubsection{第一虚拟式}


第一虚拟式的现在时使用 「第一虚拟式变位」

第一虚拟式的过去时使用 「haben/sein第一虚拟式变位 + 过去分词」

第一虚拟式的第一将来时使用 「werden第一虚拟式变位 + 不定式」

第一虚拟式的第二将来时使用 「werden第一虚拟式变位 + 过去分词 + (搭配实义动词的)haben/sein」

\subsubsection{第二虚拟式}

第二虚拟式的现在时使用 「werden第二虚拟式变位 + 不定式」

第二虚拟式的过去时使用 「haben/sein第二虚拟式变位 + 过去分词」

第二虚拟式的第一将来时使用 「werden第二虚拟式变位 + 不定式」(与现在时一样)

第二虚拟式的第二将来时使用 「werden第二虚拟式变位 + 过去分词 + (搭配实义动词的)haben/sein」

\subsection{命令式}

命令式只有现在时,根据人称使用 「命令式变位」。

不定式命令式,常见于标识或标语,有较强的“教育气息”,此时使用动词不定式,且放在句末。

建议型命令式,用于建议“一起去做...”,使用复数第二人称,使用动词不定式,且放在句末。

\subsection{被动态}

被动态主要使用 werden,有时使用 sein。

被动态的现在时、过去时使用 「werden现在时/过去时变位 + 过去分词」

被动态的完成时使用 「sein现在时变位 + 过去分词 + worden」

被动态的过去完成时使用 「sein过去时变位 + 过去分词 + worden」

被动态的第一将来时使用 「werden现在时变位 + 过去分词 + werden」

被动态的第二将来时使用 「werden现在时变位 + 过去分词 + worden + sein」

被动态的第一虚拟式使用 「werden第以虚拟式变位 + 过去分词 + werden现在时变位」

被动态的第二虚拟式使用 「werden第二虚拟式变位 + 过去分词 + sein现在时变位」

动词的过去分词可当表达被动的形容词使用,因此有时不需要构造被动态。



